
\documentclass[twocolumn]{aastex62}
\input{preface}
\newcommand{\sourcei}{SrcI\xspace}
\newcommand{\sourcen}{SrcN\xspace}
\newcommand{\sourcex}{SrcX\xspace}
\begin{document}
\input{authors}

\title{Orion \sourcei's disk is salty}
\begin{abstract}
    We report the detection of NaCl and KCl and the $^{37}$Cl and $^{41}$K
    isotopologues toward the disk around Orion \sourcei.
    This is the first detection of these molecules in the interstellar
    medium not associated with the ejecta of evolved stars.
    These lines are likely to be unique tracers of a rare process, either
    the birth of high-mass stars or the aftermath of a stellar collision.
\end{abstract}

\section{Introduction}
Molecules consisting of alkali metals and halogens have been detected
toward evolved stars including IRC +10216 \citep{Cernicharo1987a}, 
{\color{red}Discussion of AGB*s is now in the discussion session.}

\section{Observations}
The observations presented here are described in \citet{Ginsburg2018a} as part
of ALMA project 2016.1.00165.S.  We use the robust 0.5 weighted spectral cubes
from all three bands (B3 3 mm, B6 1 mm, B7 0.85 mm) for the spectral analysis
presented here.

Appendix D of that paper describes the spectral extraction method,
which we summarize here.  We used the U232.511 line (which we identify here as
NaCl v=1 J=18-17) to create a velocity centroid map.  For each spectrum with
a measured velocity, we shifted the spectrum to 0 \kms, then averaged those
spectra.  The averaging area is approximately the extent of the continuum
disk, 0.03 square arcseconds, or about 20 beam areas at band 6.

In that appendix, we reported that ``there is no consistent pattern to the
detected lines and no individual species can explain more than a few of the
observed lines.''  This statement was incorrect, as there are obvious carriers
for the majority of the unidentified lines that we had simply overlooked:
{\color{red}The following needs to be updated; I found many more transitions:}
NaCl (11 transitions in-band, 10 detections), Na$^{37}$Cl (8 transitions, 8
detections), KCl (8 transitions, 8 detections), $^{41}$KCl (8 transitions, 4
detections), and K$^{37}$Cl (7 transitions, 5 detections).
The nondetections all occur in the wings of other lines or in absorption
features from the surrounding molecular cloud.

Figures 1-3 shows the spectra with all lines labeled.

\begin{figure*}[!htp]
\includegraphics[scale=1,width=5.5in]{figures/lines_labeled_OrionSourceI_B3_spw0_robust0.5.pdf}
\includegraphics[scale=1,width=5.5in]{figures/lines_labeled_OrionSourceI_B3_spw1_robust0.5.pdf}
\includegraphics[scale=1,width=5.5in]{figures/lines_labeled_OrionSourceI_B3_spw2_robust0.5.pdf}
\includegraphics[scale=1,width=5.5in]{figures/lines_labeled_OrionSourceI_B3_spw3_robust0.5.pdf}
\caption{Stacked spectra  B3}
\label{fig:spectrab3}
\end{figure*}
\begin{figure*}[!htp]
\includegraphics[scale=1,width=5.5in]{figures/lines_labeled_OrionSourceI_B6_spw0_robust0.5.pdf}
\includegraphics[scale=1,width=5.5in]{figures/lines_labeled_OrionSourceI_B6_spw1_robust0.5.pdf}
\includegraphics[scale=1,width=5.5in]{figures/lines_labeled_OrionSourceI_B6_spw2_robust0.5.pdf}
\includegraphics[scale=1,width=5.5in]{figures/lines_labeled_OrionSourceI_B6_spw3_robust0.5.pdf}
\caption{Stacked spectra B6}
\label{fig:spectrab6}
\end{figure*}
\begin{figure*}[!htp]
\includegraphics[scale=1,width=5.5in]{figures/lines_labeled_OrionSourceI_B7.lb_spw0_robust0.5.pdf}
\includegraphics[scale=1,width=5.5in]{figures/lines_labeled_OrionSourceI_B7.lb_spw1_robust0.5.pdf}
\includegraphics[scale=1,width=5.5in]{figures/lines_labeled_OrionSourceI_B7.lb_spw2_robust0.5.pdf}
\includegraphics[scale=1,width=5.5in]{figures/lines_labeled_OrionSourceI_B7.lb_spw3_robust0.5.pdf}
\caption{Stacked spectra B7}
\label{fig:spectrab7}
\end{figure*}

\section{Results}
We have clearly identified the carrier of the majority [what number?]
of unidentified lines in the \citet{Ginsburg2018a} spectra.
The lines have amplitudes in the range 0.5-3 mJy \perbeam, corresponding
to brightness temperatures 5-20 K.

The low observed brightness temperatures suggest that the emission from
these lines is optically thin.


We plot and fit rotational diagrams to measure the abundance and temperature of
these molecules.  However, the results are ambiguous: for each vibrational
state with more than one detected transition, we are able to fit an excitation
temperature, but the abundances of the vibrationally excited states are much
higher than the ground state.  This distribution implies that the lines are
radiatively excited, and suggests that many of the unidentified transitions in
our bandpass may be from uncatalogued higher vibrational states of these same
molecules, since the abundance rises nearly linearly from v=0 to v=4 for NaCl.

The excitation temperatures derived from the rotational diagram fits are in the
range 50-100 K, which implies that the lines are sub-thermally excited.  The
radiation temperature within 50 AU of a $10^4$ \lsun star should be $T_R>500$ K,
so how do we get lower temperatures?
{\color{red} This is a place for some discussion: is it possible that the gas
is cooler and the transitions we're seeing are affected by collisions?
This is the process invoked in \citet{Kaplan2017a}, but it doesn't make sense here:
it requires the gas to be below the radiation temperature.  Also, the critical density
of these transitions is likely to be absurdly high, given their A-values in the 0.1s
range.}

\section{Discussion}
The lines we have reported have only been seen in a small handful of other sources,
all of which were moderately high-mass evolved stars blowing off their envelopes.
The known detections include the carbon star envelope IRC +10216 \citep{Cernicharo1987a},
the post-AGB star envelope CRL 2688 \citep{Highberger2003a}, and the oxygen-rich
evolved star envelopes of VY Canis Majoris and IK Tauri \citep{Milam2007a}.
\sourcei is only the 5th astronomical source in which KCl and NaCl have been
detected.
However, there were no detections of vibrationally excited NaCl or KCl in any
of the previous observations \citep[][presented the most systematic study of salt
lines, which only included detections of v=0 rotational transitions]{Agundez2012a},
while \sourcei has clear detections at least up to v=6, likely up to v=8, in both
NaCl and KCl.

In these evolved stars, v=0 rotational transitions of NaCl were observed
with rotational temperatures $T_{rot}\sim70-100$ K (IK Tau, VY CMa), inferred
local temperatures of $\sim200$ K in shocks (CRL 2688), .
% Carbon star envelope IRC +10216, d ~ 170 pc
% CRL2688 post-AGB * \citep{Highberger2003a} ~ 1 kpc
% IK Tauri, VY Canis Majoris \citep{Milam2007a}

These lines trace a rare physical phenomenon.  They are therefore an excellent tracer
of this phenomenon, assuming they are detected

Sodium and potassium are rarely observed in the dense molecular interstellar medium.
They are usually assumed to be rapidly incorporated into dust grains [cite?  Milam
says this, but there's a better original source.  Lodders \& Fegley 1999?].  Since we clearly observe NaCl
and KCl in the atmosphere of the disk, it is clear that there is a zone where either
dust has not yet formed or where dust is destroyed and returned to the gas phase.
{\color{red} What can we say about the dust destruction/sublimation zone?  I need
to do some disk research; there must be a dust sublimation zone at the surfaces
of disks in addition to the radial one.}

\subsection{Excitation}
We detect highly excited transitions of NaCl and KCl, many with Einstein A coefficients
in the range $\sim0.01-0.1$ s$^{-1}$.  These high excitation lines are unlikely
to be excited by collisions, since the gas would have to be hot enough to excite
molecules into states with $E_U > 2000$ K and would therefore likely be hot enough
to dissociate the molecules [would be nice to back this up with numbers].  Alternately,
at high densities and more moderate temperatures, the salts would likely be absorbed
into dust grains.

Within each vibrational state for which we have several transitions, we can measure
a rotational temperature.  For the KCl v=0 state, in which we have detected 4 rotational
transitions from $30 < E_U < 400$ K, the best fit rotational temperature is
$T_{rot}\sim105$ K.  At such a low temperature, the populations of the vibrationally
excited states should be effectively zero.  These observations combined suggest
that there is a radiative mechanism exciting the vibrational states.

A plausible scenario is that the hot radiation field from the central star,
with $T_*\sim4000$ K \citep{Testi2010a}, excites molecules to high vibrational
excitation states, from which they cascade through rovibrational transitions
down to the v=0 state, providing a relatively uniform level population
from $E_U=0$ to $E_U\approx T_*$ instead of a thermal level population.

\begin{table*}[htp]
\centering
\caption{NaCl Lines}
\begin{tabular}{ccccccc}
\label{tab:NaCl_salt_lines}
 J$_u$ & J$_l$ & Frequency & Velocity & Width & Amplitude & E$_U$ \\
  &  & $\mathrm{GHz}$ & $\mathrm{km\,s^{-1}}$ & $\mathrm{km\,s^{-1}}$ & $\mathrm{K}$ & $\mathrm{K}$ \\
\hline
&\vspace{-0.75em}\\
\multicolumn{7}{c}{$v = 0$} \\
\vspace{-0.75em}\\
0 & 45 & 44 & 335.05221 & 6.2 (0.4) & 6.4 (0.4) & 12.1 (0.7) & 370.4 \\
 18 & 17 & 229.24720 & 3.3 (0.1) & 4.7 (0.1) & 21.4 (0.4) & 104.6 \\
 7 & 6 & 89.22066 & 2.5 (0.4) & 5.0 (0.4) & 18.2 (1.2) & 17.1 \\
&\vspace{-0.75em}\\
\multicolumn{7}{c}{$v = 2$} \\
\vspace{-0.75em}\\
 31 & 30 & 229.68076 & 4.2 (0.6) & 13.1 (1.0) & 9.6 (0.3) & 962.5 \\
 13 & 12 & 98.70507 & 4.3 (0.9) & 7.2 (1.0) & 13.6 (1.4) & 828.3 \\
 18 & 17 & 230.77811 & 6.9 (0.1) & 6.0 (0.1) & 19.5 (0.4) & 1138.6 \\
&\vspace{-0.75em}\\
\multicolumn{7}{c}{$v = 5$} \\
\vspace{-0.75em}\\
 45 & 44 & 334.29723 & 2.7 (0.3) & 16.6 (0.6) & 42.3 (0.5) & 2332.2 \\
 31 & 30 & 230.72302 & 6.5 (0.4) & 4.3 (0.4) & 5.2 (0.5) & 2139.8 \\
&\vspace{-0.75em}\\
\multicolumn{7}{c}{$v = 6$} \\
\vspace{-0.75em}\\
 30 & 29 & 215.73578 & 7.3 (2.8) & 3.0 (2.8) & 1.9 (1.5) & 2473.0 \\
\hline
\end{tabular}

\par 
\end{table*}

\begin{table*}[htp]
\centering
\caption{Na$^{37}$Cl Lines}
\begin{tabular}{ccccccc}
\label{tab:Na37Cl_salt_lines}
 J$_u$ & J$_l$ & Frequency & Velocity & Width & Amplitude & E$_U$ \\
  &  & $\mathrm{GHz}$ & $\mathrm{km\,s^{-1}}$ & $\mathrm{km\,s^{-1}}$ & $\mathrm{K}$ & $\mathrm{K}$ \\
\hline
&\vspace{-0.75em}\\
\multicolumn{7}{c}{$v = 0$} \\
\vspace{-0.75em}\\
 7 & 6 & 89.22011 & 4.3 (0.4) & 5.0 (0.4) & 18.2 (1.2) & 17.1 \\
 18 & 17 & 229.24605 & 4.8 (0.1) & 4.7 (0.1) & 21.4 (0.4) & 104.6 \\
 17 & 16 & 214.93871 & 4.8 (0.4) & 4.6 (0.4) & 18.7 (1.2) & 607.0 \\
&\vspace{-0.75em}\\
\multicolumn{7}{c}{$v = 1$} \\
\vspace{-0.75em}\\
 8 & 7 & 101.21188 & 4.6 (0.3) & 5.7 (0.3) & 19.7 (0.8) & 536.0 \\
 8 & 7 & 100.46695 & 5.3 (0.4) & 7.9 (0.5) & 15.6 (0.7) & 1045.0 \\
&\vspace{-0.75em}\\
\multicolumn{7}{c}{$v = 2$} \\
\vspace{-0.75em}\\
 7 & 6 & 87.91232 & 4.3 (0.6) & 6.3 (0.6) & 13.1 (1.1) & 1040.1 \\
 7 & 6 & 87.26464 & 2.9 (1.0) & 4.5 (1.0) & 9.0 (1.7) & 1544.3 \\
&\vspace{-0.75em}\\
\multicolumn{7}{c}{$v = 3$} \\
\vspace{-0.75em}\\
 8 & 7 & 99.72675 & 3.7 (0.4) & 5.8 (0.4) & 14.0 (0.8) & 1549.1 \\
 7 & 6 & 86.62109 & -2.1 (1.1) & 8.1 (1.2) & 13.0 (1.4) & 2043.6 \\
&\vspace{-0.75em}\\
\multicolumn{7}{c}{$v = 4$} \\
\vspace{-0.75em}\\
 8 & 7 & 98.99127 & 4.6 (1.1) & 4.2 (1.1) & 8.2 (1.9) & 2048.4 \\
 27 & 26 & 333.45906 & 5.5 (0.5) & 4.8 (0.5) & 10.9 (1.0) & 2251.3 \\
 19 & 18 & 233.16920 & 5.2 (0.5) & 3.9 (0.5) & 7.4 (0.8) & 2633.6 \\
\hline
&\vspace{-0.75em}\\
\multicolumn{7}{c}{$v = 5$} \\
\vspace{-0.75em}\\
 7 & 6 & 85.98167 & 5.2 (1.2) & 4.9 (1.2) & 8.1 (1.7) & 2538.2 \\
\end{tabular}

\par 
\end{table*}

\begin{table*}[htp]
\centering
\caption{KCl Lines}
\begin{tabular}{ccccccc}
\label{tab:KCl_salt_lines}
 J$_u$ & J$_l$ & Frequency & Velocity & Width & Amplitude & E$_U$ \\
  &  & $\mathrm{GHz}$ & $\mathrm{km\,s^{-1}}$ & $\mathrm{km\,s^{-1}}$ & $\mathrm{K}$ & $\mathrm{K}$ \\
\hline
&\vspace{-0.75em}\\
\multicolumn{7}{c}{$v = 0$} \\
\vspace{-0.75em}\\
 28 & 27 & 215.00828 & 5.5 (0.6) & 4.3 (0.6) & 10.4 (1.3) & 149.7 \\
 30 & 29 & 230.32064 & 4.6 (0.2) & 4.8 (0.2) & 14.4 (0.4) & 171.4 \\
 45 & 44 & 344.82061 & 2.5 (1.1) & 4.9 (1.1) & 9.7 (1.8) & 381.2 \\
 13 & 12 & 99.92952 & 4.0 (0.3) & 4.8 (0.3) & 15.3 (0.9) & 33.6 \\
&\vspace{-0.75em}\\
\multicolumn{7}{c}{$v = 1$} \\
\vspace{-0.75em}\\
 44 & 43 & 335.13396 & 3.3 (0.3) & 4.1 (0.3) & 14.1 (0.8) & 761.7 \\
&\vspace{-0.75em}\\
\multicolumn{7}{c}{$v = 2$} \\
\vspace{-0.75em}\\
 13 & 12 & 98.70595 & 2.0 (0.8) & 6.4 (0.8) & 14.2 (1.5) & 828.3 \\
&\vspace{-0.75em}\\
\multicolumn{7}{c}{$v = 3$} \\
\vspace{-0.75em}\\
 13 & 12 & 98.09753 & 6.1 (1.1) & 5.0 (1.1) & 9.2 (1.7) & 1220.6 \\
 29 & 28 & 218.57971 & 4.4 (0.4) & 3.3 (0.4) & 5.0 (0.6) & 1345.1 \\
&\vspace{-0.75em}\\
\multicolumn{7}{c}{$v = 4$} \\
\vspace{-0.75em}\\
 13 & 12 & 97.49133 & 8.0 (2.1) & 7.0 (2.4) & 6.1 (1.5) & 1609.5 \\
 29 & 28 & 217.22891 & -0.3 (0.6) & 3.2 (0.6) & 3.5 (0.6) & 1733.2 \\
&\vspace{-0.75em}\\
\multicolumn{7}{c}{$v = 5$} \\
\vspace{-0.75em}\\
 31 & 30 & 230.72399 & 5.3 (0.4) & 4.3 (0.4) & 5.2 (0.5) & 2139.8 \\
 29 & 28 & 215.88373 & 6.3 (2.1) & 4.3 (2.1) & 3.0 (1.3) & 2118.0 \\
&\vspace{-0.75em}\\
\multicolumn{7}{c}{$v = 6$} \\
\vspace{-0.75em}\\
 47 & 46 & 346.87489 & 0.7 (1.1) & 5.5 (1.2) & 4.4 (0.8) & 2745.3 \\
 29 & 28 & 214.54412 & 6.3 (1.9) & 4.4 (1.9) & 3.4 (1.3) & 2499.6 \\
\hline
\end{tabular}

\par 
\end{table*}

\begin{table*}[htp]
\centering
\caption{K$^{37}$Cl Lines}
\begin{tabular}{ccccccc}
\label{tab:K37Cl_salt_lines}
 J$_u$ & J$_l$ & Frequency & Velocity & Width & Amplitude & E$_U$ \\
  &  & $\mathrm{GHz}$ & $\mathrm{km\,s^{-1}}$ & $\mathrm{km\,s^{-1}}$ & $\mathrm{K}$ & $\mathrm{K}$ \\
\hline
&\vspace{-0.75em}\\
\multicolumn{7}{c}{$v = 0$} \\
\vspace{-0.75em}\\
 45 & 44 & 335.05221 & 6.2 (0.4) & 6.4 (0.4) & 12.1 (0.7) & 370.4 \\
&\vspace{-0.75em}\\
\multicolumn{7}{c}{$v = 1$} \\
\vspace{-0.75em}\\
 12 & 11 & 89.08229 & 5.2 (1.2) & 4.7 (1.2) & 5.5 (1.2) & 421.4 \\
 31 & 30 & 229.81801 & 6.5 (0.3) & 4.8 (0.3) & 7.2 (0.4) & 570.2 \\
\hline
&\vspace{-0.75em}\\
\multicolumn{7}{c}{$v = 6$} \\
\vspace{-0.75em}\\
 30 & 29 & 215.73578 & 7.3 (2.8) & 3.0 (2.8) & 1.9 (1.5) & 2473.0 \\
\end{tabular}

\par 
\end{table*}

\begin{table*}[htp]
\centering
\caption{$^{41}$KCl Lines}
\begin{tabular}{ccccccc}
\label{tab:41KCl_salt_lines}
 J$_u$ & J$_l$ & Frequency & Velocity & Width & Amplitude & E$_U$ \\
  &  & $\mathrm{GHz}$ & $\mathrm{km\,s^{-1}}$ & $\mathrm{km\,s^{-1}}$ & $\mathrm{K}$ & $\mathrm{K}$ \\
\hline
&\vspace{-0.75em}\\
\multicolumn{7}{c}{$v = 0$} \\
\vspace{-0.75em}\\
 29 & 28 & 217.54351 & 5.0 (0.7) & 6.5 (0.7) & 4.5 (0.4) & 156.7 \\
 13 & 12 & 97.62803 & 4.9 (2.5) & 10.0 (3.2) & 6.5 (1.3) & 32.8 \\
&\vspace{-0.75em}\\
\multicolumn{7}{c}{$v = 2$} \\
\vspace{-0.75em}\\
 12 & 11 & 89.03049 & 8.2 (0.0) & 0.2 (0.0) & 0.1 (1.6) & 813.8 \\
\hline
\end{tabular}

\par 
\end{table*}

\begin{table*}[htp]
\centering
\caption{$^{41}$K$^{37}$Cl Lines}
\begin{tabular}{ccccccc}
\label{tab:41K37Cl_salt_lines}
 J$_u$ & J$_l$ & Frequency & Velocity & Width & Amplitude & E$_U$ \\
  &  & $\mathrm{GHz}$ & $\mathrm{km\,s^{-1}}$ & $\mathrm{km\,s^{-1}}$ & $\mathrm{K}$ & $\mathrm{K}$ \\
\hline
\hline
&\vspace{-0.75em}\\
\multicolumn{7}{c}{$v = 7$} \\
\vspace{-0.75em}\\
7 & 48 & 47 & 334.32520 & -9.5 (0.0) & 7.1 (0.3) & 27.8 (0.8) & 3049.3 \\
\end{tabular}

\par 
\end{table*}


\input{solobib}

\end{document}
