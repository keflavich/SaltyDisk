
\documentclass[twocolumn]{aastex62}
\input{preface}
\newcommand{\sourcei}{SrcI\xspace}
\newcommand{\sourcen}{SrcN\xspace}
\newcommand{\sourcex}{SrcX\xspace}
\newcommand{\bam}[1]{\textcolor{green!65!black}{\textbf{[BAM: #1]}}}
\begin{document}
\input{authors}

\title{Orion \sourcei's disk is salty}
\begin{abstract}
    We report the detection of NaCl and KCl and their $^{37}$Cl and $^{41}$K
    isotopologues toward the disk around Orion \sourcei.
    This is the first detection of these molecules in the interstellar
    medium not associated with the ejecta of evolved stars.  It is also
    the first ever detection of the vibrationally excited states of these
    lines in the ISM, up to v=8.
    These lines are likely to be unique tracers of a rare process, either
    the birth of high-mass stars or the aftermath of a stellar collision.
\end{abstract}

\section{Introduction}
Molecules consisting of alkali metals and halogens have been detected
toward evolved stars including the carbon star IRC +10216 \citep{Cernicharo1987a},
oxygen-rich evolved stars IK Tauri and VY Canis Majoris \citep{Milam2007a},
and the post-AGB star envelope CRL 2688 \citep{Highberger2003a}.
The limited environments in which these molecules have been detected
suggests that they only exist in the gas phase for a brief period
when being ejected from dying stars before being incorporated into
dust grains.

The lines of NaCl and KCl have been catalogued by \citet{Caris2004a} and more
comprehensively by \citet{Barton2014a}.  The latter includes over 10$^5$
transitions for each species with levels up to $E_U\sim5\ee{4}$ K.  Prior to
this work, no astronomical detections of vibrationally excited transitions of
any of these species have been reported \citep[e.g., see the review
by][]{McGuire2018a}. \bam{Did we check w/ Holger here to see where he thought the v=1 for NaCl was seen?  I still can't find a paper w/ it.  It's also probably not correct to cite my paper here - I haven't (yet) compiled vibrational information...}

Disks around forming high-mass stars have been resolved only a handful of
times.  The disk around Orion Source I (\sourcei) has frequently been observed
\citep{Hirota2014a,Plambeck2016a,Ginsburg2018a}, but most molecular species
detected trace either an outflow (e.g., SiO) or some combination of the outflow
and the disk (e.g., \water).  In HH80/81, a $\sim4$ \msun disk around a
$\sim10-20$ \msun star, several lines of SO$_2$ are associated with the disk
\citep{Girart2017a}.  Beyond this pair, only disordered structures or
disk-envelope blends have been observed. \bam{In HH80/81, or do you mean beyond this pair only disordered blends have been observed period, anywhere?}

In \citet[][the brine paper]{Ginsburg2018b}, we used lines of \water and one or
more unknown species to measure the rotation curve of the disk around \sourcei.
We have now identified these unknown lines as transitions of NaCl and KCl.

\section{Observations and Analysis}
The observations presented here are described in \citet{Ginsburg2018b} as part
of ALMA project 2016.1.00165.S.  We use the robust 0.5 weighted spectral cubes
from all three bands (B3 3 mm, B6 1 mm, B7 0.85 mm) for the spectral analysis
presented here.

Appendix D of that paper describes the spectral extraction method,
which we summarize here.  We used the U232.511 line (which we identify here as
NaCl v=1 J=18-17) to create a velocity centroid map.  For each spectrum with
a measured velocity, we shifted the spectrum to 0 \kms, then averaged those
spectra.  The averaging area is approximately the extent of the continuum
disk, 0.03 square arcseconds, or about 20 beam areas at band 6, resulting
in an improvement in the signal-to-noise of about $4\times$.
This stacked spectrum, shown in Figures \ref{fig:spectrab3}-\ref{fig:spectrab7},
is the average spectrum across the disk identified in \citet{Ginsburg2018b}:
it traces material just above and below the optically thick continuum disk, and
for the lines discussed here, includes no emission associated with the outflow.


We have clearly identified the carrier of the majority of unidentified lines in
the \citet{Ginsburg2018b} spectra (only 2-3 lines are unidentified in the B6/B7
spectra, though there are several in B3 we have not identified). \bam{Can you pass those along?  Might be worth taking another look just in case.}  The lines
have amplitudes in the range 0.5-3 mJy \perbeam, corresponding
to brightness temperatures 5-20 K.

In the appendix of \citet{Ginsburg2018b}, we reported that ``there is no
consistent pattern to the detected lines and no individual species can explain
more than a few of the observed lines.''  This statement was incorrect, as
there are obvious carriers for the majority of the unidentified lines that we
had simply overlooked: NaCl, KCl, and their isotopologues.

Figures \ref{fig:spectrab3}-\ref{fig:spectrab7} shows the spectra with lines labeled.  We have labeled all of the
detections and marginal detections of salt lines.  We also labeled the most
prominent outflow (e.g., SiO and H$_2$O) lines.
%We additionally labeled
%\emph{some} of the lines that were not detected because of severe confusion

\begin{figure*}[!htp]
\includegraphics[scale=1,width=5.5in]{figures/lines_labeled_OrionSourceI_B3_spw0_robust0.5.pdf}
\includegraphics[scale=1,width=5.5in]{figures/lines_labeled_OrionSourceI_B3_spw1_robust0.5.pdf}
\includegraphics[scale=1,width=5.5in]{figures/lines_labeled_OrionSourceI_B3_spw2_robust0.5.pdf}
\includegraphics[scale=1,width=5.5in]{figures/lines_labeled_OrionSourceI_B3_spw3_robust0.5.pdf}
\caption{Stacked spectra  B3}
\label{fig:spectrab3}
\end{figure*}
\begin{figure*}[!htp]
\includegraphics[scale=1,width=5.5in]{figures/lines_labeled_OrionSourceI_B6_spw0_robust0.5.pdf}
\includegraphics[scale=1,width=5.5in]{figures/lines_labeled_OrionSourceI_B6_spw1_robust0.5.pdf}
\includegraphics[scale=1,width=5.5in]{figures/lines_labeled_OrionSourceI_B6_spw2_robust0.5.pdf}
\includegraphics[scale=1,width=5.5in]{figures/lines_labeled_OrionSourceI_B6_spw3_robust0.5.pdf}
\caption{Stacked spectra B6}
\label{fig:spectrab6}
\end{figure*}
\begin{figure*}[!htp]
\includegraphics[scale=1,width=5.5in]{figures/lines_labeled_OrionSourceI_B7.lb_spw0_robust0.5.pdf}
\includegraphics[scale=1,width=5.5in]{figures/lines_labeled_OrionSourceI_B7.lb_spw1_robust0.5.pdf}
\includegraphics[scale=1,width=5.5in]{figures/lines_labeled_OrionSourceI_B7.lb_spw2_robust0.5.pdf}
\includegraphics[scale=1,width=5.5in]{figures/lines_labeled_OrionSourceI_B7.lb_spw3_robust0.5.pdf}
\caption{Stacked spectra B7}
\label{fig:spectrab7}
\end{figure*}


In Tables \ref{tab:all_detections_B3}-\ref{tab:all_detections_B7}, we report
all of the lines that are in our observable band and whether or not they were
detected.  We employ the following
flag scheme:
`d' for detection `n' for nondetection, `q' for questionable (e.g., low
signal-to-noise, but possibly detected, or in regions where the noise is
not Gaussian and may be affected by other lines).
We additionally include a flag `c' for `confused', which we add to the flag
string if the line is blended with another brighter line of a different species.

We use the \citet{Barton2014a} catalog to identify lines with vibrational
states $v>2$ for KCl and $v>4$ for NaCl, since these were not cataloged in any
of the Splatalogue sources (i.e., CDMS, SLAIM, JPL {\color{red}TODO: cite
these}).   However, we found a systematic discrepancy in the rest frequencies
reported by \citet{Barton2014a} and the other catalogs, with the KCl lines
being offset by typically 15-20 \kms.  The offset is a function of the vibrational
and rotational energy states, indicating that it resulted from an incorrect rotational
and distortion constant.  We correct for this error by performing a bilinear fit
in frequency as a function of $v_u$ and $J_u$.  In the transitions available
in both catalogs, the resulting offsets are generally $<0.1$ \kms, which is
good enough to get a reliable match.  We apply these fitted models to the
higher-$v$ states to get corrected rest frequencies.

% \textbf{\color{red} At the moment, all of the line frequencies are shifted
% to higher frequencies by 17 \kms (KCl) and 3 \kms (NaCl) because of a discrepancy
% between the \citep{Barton2014a} catalog line frequencies and those of CDMS,
% JPL, and SLAIM.  The latter three, catalogued in Splatalogue, agree, and they
% agree better with my measurements.}

\begin{table*}[htp]
\centering
\caption{All detected lines in Band 3}
\begin{tabular}{ccccccc}
\label{tab:all_detections_B3}
Species & v & J$_u$ & J$_l$ & E$_U$ & Frequency & Flag \\
\hline
$^{41}$K$^{35}$Cl & 8 & 12 & 11 & 3092.4 & 85.80766 & -n \\
$^{23}$Na$^{35}$Cl & 8 & 7 & 6 & 4031.9 & 85.86278 & -n \\
$^{39}$K$^{37}$Cl & 7 & 12 & 11 & 2713.3 & 85.87379 & -n \\
$^{41}$K$^{37}$Cl & 3 & 12 & 11 & 1183.9 & 85.93776 & -n \\
$^{23}$Na$^{37}$Cl & 5 & 7 & 6 & 2536.0 & 85.97809 & -d \\
$^{41}$K$^{35}$Cl & 7 & 12 & 11 & 2720.7 & 86.33986 & cn \\
$^{39}$K$^{37}$Cl & 6 & 12 & 11 & 2339.4 & 86.40364 & -n \\
$^{41}$K$^{37}$Cl & 2 & 12 & 11 & 801.6 & 86.45668 & -n \\
$^{23}$Na$^{35}$Cl & 7 & 7 & 6 & 3547.0 & 86.51213 & -q \\
$^{23}$Na$^{37}$Cl & 4 & 7 & 6 & 2041.9 & 86.61903 & -d \\
$^{39}$K$^{35}$Cl & 10 & 12 & 11 & 3869.7 & 86.68041 & -n \\
$^{41}$K$^{35}$Cl & 6 & 12 & 11 & 2345.7 & 86.87413 & -n \\
$^{39}$K$^{37}$Cl & 5 & 12 & 11 & 1962.2 & 86.93553 & -n \\
$^{41}$K$^{37}$Cl & 1 & 12 & 11 & 416.1 & 86.97743 & -q \\
$^{23}$Na$^{35}$Cl & 6 & 7 & 6 & 3057.4 & 87.16544 & -d \\
$^{39}$K$^{35}$Cl & 9 & 12 & 11 & 3500.3 & 87.22792 & -n \\
$^{23}$Na$^{37}$Cl & 3 & 7 & 6 & 1543.0 & 87.26372 & -d \\
$^{41}$K$^{35}$Cl & 5 & 12 & 11 & 1967.6 & 87.41041 & -n \\
$^{39}$K$^{37}$Cl & 4 & 12 & 11 & 1581.9 & 87.46938 & -n \\
$^{41}$K$^{37}$Cl & 0 & 12 & 11 & 27.3 & 87.50010 & -q \\
$^{39}$K$^{35}$Cl & 8 & 12 & 11 & 3127.7 & 87.77742 & -n \\
$^{23}$Na$^{35}$Cl & 5 & 7 & 6 & 2563.0 & 87.82283 & -d \\
$^{23}$Na$^{37}$Cl & 2 & 7 & 6 & 1039.3 & 87.91214 & -d \\
$^{41}$K$^{35}$Cl & 4 & 12 & 11 & 1586.3 & 87.94866 & -n \\
$^{39}$K$^{37}$Cl & 3 & 12 & 11 & 1198.3 & 88.00523 & -d \\
$^{39}$K$^{35}$Cl & 7 & 12 & 11 & 2751.8 & 88.32895 & -q \\
$^{23}$Na$^{35}$Cl & 4 & 7 & 6 & 2063.8 & 88.48417 & -d \\
$^{41}$K$^{35}$Cl & 3 & 12 & 11 & 1201.7 & 88.48896 & cq \\
$^{39}$K$^{37}$Cl & 2 & 12 & 11 & 811.5 & 88.54307 & -d \\
$^{23}$Na$^{37}$Cl & 1 & 7 & 6 & 530.7 & 88.56426 & -d \\
$^{39}$K$^{35}$Cl & 6 & 12 & 11 & 2372.8 & 88.88253 & -n \\
$^{41}$K$^{35}$Cl & 2 & 12 & 11 & 813.8 & 89.03129 & -d \\
$^{39}$K$^{37}$Cl & 1 & 12 & 11 & 421.4 & 89.08292 & -d \\
$^{23}$Na$^{35}$Cl & 3 & 7 & 6 & 1559.6 & 89.14960 & -d \\
$^{41}$K$^{37}$Cl & 10 & 13 & 12 & 3776.8 & 89.21871 & -n \\
$^{23}$Na$^{37}$Cl & 0 & 7 & 6 & 17.1 & 89.22014 & -d \\
$^{39}$K$^{35}$Cl & 4 & 13 & 12 & 1609.5 & 97.49133 & -d \\
$^{23}$Na$^{37}$Cl & 6 & 8 & 7 & 3030.0 & 97.52883 & -q \\
$^{41}$K$^{35}$Cl & 0 & 13 & 12 & 32.8 & 97.62809 & -d \\
$^{41}$K$^{37}$Cl & 7 & 14 & 13 & 2690.7 & 97.85364 & -n \\
$^{39}$K$^{35}$Cl & 3 & 13 & 12 & 1220.6 & 98.09753 & -d \\
$^{23}$Na$^{35}$Cl & 8 & 8 & 7 & 4036.6 & 98.12538 & -q \\
$^{23}$Na$^{37}$Cl & 5 & 8 & 7 & 2540.7 & 98.25682 & -q \\
$^{39}$K$^{37}$Cl & 10 & 14 & 13 & 3825.5 & 98.33904 & -n \\
$^{41}$K$^{37}$Cl & 6 & 14 & 13 & 2321.0 & 98.44993 & -n \\
$^{39}$K$^{35}$Cl & 2 & 13 & 12 & 828.3 & 98.70595 & -d \\
$^{23}$Na$^{35}$Cl & 7 & 8 & 7 & 3551.7 & 98.86719 & -q \\
$^{41}$K$^{35}$Cl & 10 & 14 & 13 & 3835.7 & 98.86759 & cn \\
$^{39}$K$^{37}$Cl & 9 & 14 & 13 & 3461.0 & 98.94969 & -n \\
$^{23}$Na$^{37}$Cl & 4 & 8 & 7 & 2046.6 & 98.98914 & -d \\
$^{41}$K$^{37}$Cl & 5 & 14 & 13 & 1948.2 & 99.04845 & -n \\
$^{39}$K$^{35}$Cl & 1 & 13 & 12 & 432.6 & 99.31663 & -d \\
$^{41}$K$^{35}$Cl & 9 & 14 & 13 & 3470.3 & 99.48329 & -n \\
$^{39}$K$^{37}$Cl & 8 & 14 & 13 & 3093.3 & 99.56276 & -n \\
$^{23}$Na$^{35}$Cl & 6 & 8 & 7 & 3062.2 & 99.61369 & -d \\
$^{41}$K$^{37}$Cl & 4 & 14 & 13 & 1572.3 & 99.64924 & -n \\
$^{23}$Na$^{37}$Cl & 3 & 8 & 7 & 1547.8 & 99.72579 & -d \\
$^{39}$K$^{35}$Cl & 0 & 13 & 12 & 33.6 & 99.92952 & -d \\
$^{41}$K$^{35}$Cl & 8 & 14 & 13 & 3101.7 & 100.10138 & -n \\
$^{39}$K$^{37}$Cl & 7 & 14 & 13 & 2722.6 & 100.17822 & -n \\
$^{41}$K$^{37}$Cl & 3 & 14 & 13 & 1193.2 & 100.25229 & -n \\
$^{23}$Na$^{35}$Cl & 5 & 8 & 7 & 2567.8 & 100.36474 & -d \\
$^{23}$Na$^{37}$Cl & 2 & 8 & 7 & 1044.1 & 100.46673 & -d \\
$^{41}$K$^{35}$Cl & 7 & 14 & 13 & 2730.0 & 100.72190 & -n \\
$^{39}$K$^{37}$Cl & 6 & 14 & 13 & 2348.7 & 100.79603 & -q \\
$^{41}$K$^{37}$Cl & 2 & 14 & 13 & 810.9 & 100.85758 & -n \\
$^{23}$Na$^{35}$Cl & 4 & 8 & 7 & 2068.6 & 101.12050 & -d \\
$^{39}$K$^{35}$Cl & 10 & 14 & 13 & 3879.0 & 101.12088 & -n \\
$^{23}$Na$^{37}$Cl & 1 & 8 & 7 & 535.5 & 101.21200 & -d \\
\hline
\end{tabular}

\par 
\end{table*}

\begin{table*}[htp]
\centering
\caption{All detected lines in Band 6}
\begin{tabular}{ccccccc}
\label{tab:all_detections_B6}
Species & v & J$_u$ & J$_l$ & E$_U$ & Frequency & Flag \\
\hline
$^{39}$K$^{37}$Cl & 7 & 30 & 29 & 2846.1 & 214.41527 & -n \\
$^{23}$Na$^{37}$Cl & 9 & 18 & 17 & 4547.4 & 214.43575 & cn \\
$^{39}$K$^{35}$Cl & 6 & 29 & 28 & 2499.6 & 214.54412 & -d \\
$^{41}$K$^{37}$Cl & 3 & 30 & 29 & 1316.8 & 214.57918 & -n \\
$^{23}$Na$^{35}$Cl & 4 & 17 & 16 & 2139.6 & 214.74036 & cn \\
$^{41}$K$^{35}$Cl & 2 & 29 & 28 & 940.8 & 214.90740 & cq \\
$^{23}$Na$^{37}$Cl & 1 & 17 & 16 & 606.5 & 214.93872 & -d \\
$^{39}$K$^{35}$Cl & 0 & 28 & 27 & 149.7 & 215.00828 & -d \\
$^{39}$K$^{37}$Cl & 1 & 29 & 28 & 548.5 & 215.03464 & cq \\
$^{41}$K$^{37}$Cl & 8 & 31 & 30 & 3187.6 & 215.09368 & cn \\
$^{41}$K$^{35}$Cl & 7 & 30 & 29 & 2854.2 & 215.57755 & cn \\
$^{39}$K$^{37}$Cl & 6 & 30 & 29 & 2473.0 & 215.73679 & -d \\
$^{41}$K$^{37}$Cl & 2 & 30 & 29 & 935.3 & 215.87559 & -n \\
$^{39}$K$^{35}$Cl & 5 & 29 & 28 & 2118.0 & 215.88373 & -d \\
$^{23}$Na$^{37}$Cl & 8 & 18 & 17 & 4072.5 & 216.04057 & -n \\
$^{39}$K$^{37}$Cl & 5 & 30 & 29 & 2096.7 & 217.06391 & -q \\
$^{41}$K$^{37}$Cl & 1 & 30 & 29 & 550.6 & 217.17723 & -q \\
$^{39}$K$^{35}$Cl & 4 & 29 & 28 & 1733.2 & 217.22891 & cd \\
$^{23}$Na$^{35}$Cl & 10 & 18 & 17 & 5070.5 & 217.31725 & cn \\
$^{39}$K$^{37}$Cl & 10 & 31 & 30 & 3957.2 & 217.47744 & -n \\
$^{41}$K$^{35}$Cl & 0 & 29 & 28 & 156.7 & 217.54317 & -d \\
$^{23}$Na$^{37}$Cl & 7 & 18 & 17 & 3593.1 & 217.65560 & -q \\
$^{41}$K$^{37}$Cl & 6 & 31 & 30 & 2452.9 & 217.72366 & -n \\
$^{39}$K$^{35}$Cl & 9 & 30 & 29 & 3635.2 & 217.79704 & cn \\
$^{23}$Na$^{35}$Cl & 2 & 17 & 16 & 1127.5 & 217.97998 & -d \\
$^{41}$K$^{35}$Cl & 5 & 30 & 29 & 2102.8 & 218.24805 & -n \\
$^{39}$K$^{37}$Cl & 4 & 30 & 29 & 1717.2 & 218.39657 & cn \\
$^{41}$K$^{37}$Cl & 0 & 30 & 29 & 162.6 & 218.48414 & cn \\
$^{39}$K$^{35}$Cl & 3 & 29 & 28 & 1345.1 & 218.57971 & -d \\
$^{41}$K$^{35}$Cl & 10 & 31 & 30 & 3968.1 & 218.64471 & -n \\
$^{39}$K$^{37}$Cl & 9 & 31 & 30 & 3593.5 & 218.82568 & -q \\
$^{23}$Na$^{37}$Cl & 0 & 18 & 17 & 104.6 & 229.24601 & -d \\
$^{39}$K$^{35}$Cl & 6 & 31 & 30 & 2521.2 & 229.29217 & cd \\
$^{23}$Na$^{35}$Cl & 10 & 19 & 18 & 5081.5 & 229.36540 & cn \\
$^{41}$K$^{35}$Cl & 2 & 31 & 30 & 962.5 & 229.68227 & cd \\
$^{23}$Na$^{37}$Cl & 7 & 19 & 18 & 3604.1 & 229.72316 & cn \\
$^{39}$K$^{37}$Cl & 1 & 31 & 30 & 570.2 & 229.81880 & -d \\
$^{41}$K$^{35}$Cl & 7 & 32 & 31 & 2875.9 & 229.90046 & -q \\
$^{39}$K$^{37}$Cl & 6 & 32 & 31 & 2494.7 & 230.07072 & -d \\
$^{41}$K$^{37}$Cl & 2 & 32 & 31 & 957.1 & 230.22070 & -q \\
$^{41}$K$^{37}$Cl & 7 & 33 & 32 & 2843.6 & 230.31852 & cn \\
$^{39}$K$^{35}$Cl & 0 & 30 & 29 & 171.4 & 230.32064 & -d \\
$^{39}$K$^{35}$Cl & 5 & 31 & 30 & 2139.8 & 230.72399 & -d \\
$^{23}$Na$^{35}$Cl & 2 & 18 & 17 & 1138.6 & 230.77883 & -d \\
$^{39}$K$^{35}$Cl & 10 & 32 & 31 & 4025.6 & 230.81309 & -n \\
$^{39}$K$^{35}$Cl & 4 & 31 & 30 & 1755.1 & 232.16185 & -d \\
$^{39}$K$^{35}$Cl & 9 & 32 & 31 & 3657.1 & 232.26613 & cn \\
$^{41}$K$^{35}$Cl & 0 & 31 & 30 & 178.7 & 232.49980 & cn \\
$^{23}$Na$^{35}$Cl & 1 & 18 & 17 & 625.2 & 232.50998 & -d \\
$^{41}$K$^{35}$Cl & 10 & 33 & 32 & 3990.1 & 232.69933 & cn \\
$^{41}$K$^{35}$Cl & 5 & 32 & 31 & 2124.8 & 232.74872 & -q \\
$^{23}$Na$^{35}$Cl & 8 & 19 & 18 & 4127.2 & 232.84920 & -q \\
$^{39}$K$^{37}$Cl & 9 & 33 & 32 & 3615.5 & 232.89230 & -n \\
$^{39}$K$^{37}$Cl & 4 & 32 & 31 & 1739.2 & 232.90755 & -d \\
$^{41}$K$^{37}$Cl & 10 & 34 & 33 & 3942.7 & 232.97243 & cn \\
$^{41}$K$^{37}$Cl & 0 & 32 & 31 & 184.6 & 233.00319 & cn \\
$^{41}$K$^{37}$Cl & 5 & 33 & 32 & 2102.9 & 233.12954 & cn \\
$^{23}$Na$^{37}$Cl & 5 & 19 & 18 & 2631.4 & 233.16458 & cd \\
$^{39}$K$^{35}$Cl & 3 & 31 & 30 & 1367.1 & 233.60570 & -d \\
$^{39}$K$^{35}$Cl & 8 & 32 & 31 & 3285.5 & 233.72540 & -n \\
\hline
\end{tabular}

\par 
\end{table*}

\begin{table*}[htp]
\centering
\caption{All detected lines in Band 7}
\begin{tabular}{ccccccc}
\label{tab:all_detections_B7}
Species & v & J$_u$ & J$_l$ & E$_U$ & Frequency & Flag \\
 &  &  &  & $\mathrm{K}$ & $\mathrm{GHz}$ &  \\
\hline
$^{39}$K$^{37}$Cl & 5 & 46 & 45 & 2310.3 & 332.14133 & cn \\
$^{41}$K$^{37}$Cl & 1 & 46 & 45 & 764.4 & 332.34799 & -n \\
$^{41}$K$^{35}$Cl & 2 & 45 & 44 & 1154.0 & 332.81245 & -n \\
$^{23}$Na$^{35}$Cl & 2 & 26 & 25 & 1249.3 & 333.00524 & -d \\
$^{39}$K$^{37}$Cl & 1 & 45 & 44 & 761.8 & 333.01818 & cn \\
$^{39}$K$^{35}$Cl & 2 & 44 & 43 & 1155.3 & 333.06642 & -d \\
$^{41}$K$^{37}$Cl & 4 & 47 & 46 & 1921.1 & 333.42706 & -n \\
$^{23}$Na$^{37}$Cl & 4 & 27 & 26 & 2249.5 & 333.45280 & -d \\
$^{41}$K$^{35}$Cl & 5 & 46 & 45 & 2317.6 & 333.94969 & -n \\
$^{39}$K$^{37}$Cl & 4 & 46 & 45 & 1932.1 & 334.18411 & -n \\
$^{39}$K$^{35}$Cl & 5 & 45 & 44 & 2332.2 & 334.29723 & -d \\
$^{41}$K$^{37}$Cl & 0 & 46 & 45 & 377.7 & 334.35365 & cn \\
$^{41}$K$^{35}$Cl & 1 & 45 & 44 & 764.9 & 334.85440 & cq \\
$^{39}$K$^{37}$Cl & 0 & 45 & 44 & 370.4 & 335.05221 & -d \\
$^{39}$K$^{35}$Cl & 1 & 44 & 43 & 761.7 & 335.13388 & -d \\
$^{41}$K$^{37}$Cl & 3 & 47 & 46 & 1544.2 & 335.44973 & -n \\
$^{23}$Na$^{35}$Cl & 1 & 26 & 25 & 736.8 & 335.50625 & -d \\
$^{41}$K$^{35}$Cl & 0 & 46 & 45 & 389.0 & 344.34007 & cn \\
$^{41}$K$^{37}$Cl & 2 & 48 & 47 & 1180.5 & 344.60831 & cq \\
$^{39}$K$^{35}$Cl & 0 & 45 & 44 & 381.2 & 344.82231 & -d \\
$^{41}$K$^{35}$Cl & 3 & 47 & 46 & 1572.6 & 345.37333 & cn \\
$^{41}$K$^{37}$Cl & 5 & 49 & 48 & 2327.8 & 345.40571 & cn \\
$^{39}$K$^{37}$Cl & 2 & 47 & 46 & 1182.6 & 345.59753 & cn \\
$^{23}$Na$^{37}$Cl & 4 & 28 & 27 & 2266.1 & 345.74820 & cn \\
$^{23}$Na$^{35}$Cl & 2 & 27 & 26 & 1265.9 & 345.75983 & cd \\
$^{39}$K$^{35}$Cl & 3 & 46 & 45 & 1578.4 & 345.94612 & cn \\
$^{39}$K$^{37}$Cl & 5 & 48 & 47 & 2343.3 & 346.47135 & -n \\
$^{41}$K$^{37}$Cl & 1 & 48 & 47 & 797.3 & 346.69213 & cn \\
$^{41}$K$^{35}$Cl & 2 & 47 & 46 & 1187.0 & 347.49658 & -q \\
$^{41}$K$^{37}$Cl & 4 & 49 & 48 & 1954.1 & 347.50516 & -n \\
$^{39}$K$^{37}$Cl & 1 & 47 & 46 & 794.8 & 347.71265 & -d \\
\hline
\end{tabular}

\par 
\end{table*}


%The low observed brightness temperatures suggest that the emission from
%these lines is optically thin.




\section{Discussion}
The lines we have reported have only been seen in a small handful of other sources,
all of which were moderately high-mass evolved stars blowing off their envelopes.
The known detections include the carbon star envelope IRC +10216 \citep{Cernicharo1987a},
the post-AGB star envelope CRL 2688 \citep{Highberger2003a}, and the oxygen-rich
evolved star envelopes of VY Canis Majoris and IK Tauri \citep{Milam2007a}.
\sourcei is only the 5th astronomical source in which KCl and NaCl have been
detected.
However, there were no detections of vibrationally excited NaCl or KCl in any
of the previous observations (\citealt{Agundez2012a} presented the most
systematic study of salt lines, which only included detections of v=0
rotational transitions),
while \sourcei exhibits clear emission up to v=6 in both NaCl and KCl.

In these evolved stars, v=0 rotational transitions of NaCl were observed
with rotational temperatures ${T_{rot}\sim70-100}$ K (IK Tau, VY CMa), inferred
local temperatures of $\sim200$ K in shocks (CRL 2688) and $T\approx700$ K  in
an expanding shell (IRC+10216).
% Carbon star envelope IRC +10216, d ~ 170 pc
% CRL2688 post-AGB * \citep{Highberger2003a} ~ 1 kpc
% IK Tauri, VY Canis Majoris \citep{Milam2007a}

In \sourcei, the emission comes from the surface layers of the disk.
\citet{Ginsburg2018a} showed that these species come only from the disk's
immediate surroundings, unlike SiO and \water, which exhibit high vertical
extents consistent with outflow.  They modeled the NaCl emission profile
as a truncated Keplerian disk, finding an inner cutoff $\approx$35-40 AU
and an outer cutoff $\approx55-60$ AU, indicating that the salt-emitting
region is a narrow portion of the disk.

% ((constants.sigma_sb * 4 * np.pi * (35*u.au)**2 / (1e4*u.L_sun))**(-1/4)).decompose()
% <Quantity 665.33595877 K>
% ((constants.sigma_sb * 4 * np.pi * (60*u.au)**2 / (1e4*u.L_sun))**(-1/4)).decompose()
% <Quantity 508.15873227 K>
The equilibrium temperature in this range can be computed assuming the central
source has a luminosity $L=10^4 \lsun = 4 \pi r^2 \sigma_{SB} T^4 $, where we
solve for temperature with $r$ at the inner and outer radius, giving $T_{eq} =
508-665$ K for $r=35-60$ AU, assuming the disk intercepts all of the starlight 
at this radius.  More realistically, providing the opposite limiting case,
for a flat disk, $T_{eq}=120-185$ K \citep{Chiang1997a}.  We discuss
these temperatures below.
% This estimate provides the expected temperature of
% gas at the surface of a passive disk in this radius range. {\color{red}We could
% obtain more realistic temperature estimates using the \citet{Chiang1997a}
% approximations for a flat or flared disk.}

\subsection{Excitation}
% We detect highly excited transitions of NaCl and KCl, many with Einstein A
% coefficients in the range $\sim0.01-0.1$ s$^{-1}$, suggesting they are
% radiatively rather than collisionally excited.

% These high excitation lines
% are unlikely to be excited by collisions, since the gas would have to be hot
% enough to excite molecules into states with $E_U > 2000$ K and would therefore
% likely be hot enough to dissociate the molecules {\color{red} It would be nice
% to back this up with numbers}.  Alternately, at high densities and more
% moderate temperatures, the salts would likely be absorbed into dust grains.

% We calculate the level population as
% \begin{equation}
%     N_u = \frac{8 \pi \nu k_B}{g_u c^2 h A_{ul}} \int T_A d\nu
% %    nline = 8 * np.pi * freq * constants.k_B / constants.h / Aul / constants.c**2
% %    # term2 = np.exp(-constants.h*freq/(constants.k_B*Tex)) -1
% %    # term2 -> kt / hnu
% %    # kelvin-hertz
% %    Khz = (kkms * (freq/constants.c)).to(u.K * u.MHz)
% %    return (nline * Khz / degeneracies).to(u.cm**-2)
% \end{equation}
% {\color{red} (this is mostly a sanity check)}

Within each vibrational state for which we have several transitions, we attempt
to measure a rotational temperature.  For the KCl v=0 state, in which we have
detected 4 rotational transitions from $30 < E_U < 400$ K, the best fit
rotational temperature is $T_{rot}\sim105$ K (Figure
\ref{fig:rotationdiagrams}).  At such a low temperature, the populations of the
vibrationally excited states should be effectively zero.  These observations
combined suggest that there is a radiative mechanism exciting the vibrational
states.

The rotational temperatures are reasonably consistent with the lower bound
disk temperatures at this radius, hinting that the salt emission lines are
coming from somewhere below the surface of the disk.  If this is the case,
there are likely to be few optical photons reaching the salt molecules,
since those are readily absorbed by the small grains in the upper layers.


% Since the highest clearly detected energy state detected has $E_U\sim3000$ K
% (see Table \ref{tab:all_detections_B3}; the v=6 lines of NaCl have three
% J-transitions detected), the radiation field likely turns over at around this
% temperature, implying the presence of a slightly cooler radiation field than
% implied in \citet{Testi2010a}, but still much warmer than the equilibrium temperature
% of the disk at this radius.  
% However, it is also possible that the higher-excitation lines are simply too faint
% to be detected at our $\sim0.4$ mJy/\kms RMS limit.

%{\color{red} This is a note to be incorporated somewhere:}
The $\Delta v=1$ rovibrational transitions of KCl are all around 36.1-38 \um,
while the NaCl transitions are in the range 27.6-30 \um.
More interesting for constraining the exciting source, though, are the v=0 to
v=1 through v=6 lines.
For example, the KCl v=6-0 J=1-0 line is at 6.13 \um and the NaCl v=7-0 J=1-0
line is at 4.07 \um.  \bam{These would be wickedly forbidden transitions.  Allowed vibrational transitions have $\Delta$V = $\pm$1.  Are there better candidates that would follow that?  Say a v$_6$ - v$_5$ or something?} Assuming the population of these lines are created
from excitation from the ground state, they require a radiation field with
significant emission in the 4-6 \um range.
A blackbody with f $T\approx500-750$ K
% ($T\approx b / \lambda$, where $b\approx2900$ \um K is Wien's displacement constant)
peaks in this range.


A plausible scenario is that the hot radiation field from the central star,
with $T_*\sim4000$ K \citep{Testi2010a}, excites molecules to high vibrational
excitation states, from which they cascade through rovibrational transitions
down to the v=0 state, providing a relatively uniform level population
across the excited vibrational states instead of a thermal level population.
However, a 4000 K blackbody should produce most of its flux around 1 \um,
which should result in much higher excitation than we observe, suggesting
that the starlight is extincted before reaching the salts.

The densities required to produce equilibrium populations within the individual
vibrationally excited states are very high.  The Einstein A coefficients of the
$\Delta v=1$ transitions are of order $\sim1$ \pers.  If we assume the
collision rates for the salts are similar to SiS, the most similar-mass
two-atom molecule available from
\url{http://home.strw.leidenuniv.nl/~moldata/datafiles}, with values
$\sim10^{-12}-10^{-10}$ cm$^{3}$ \pers, densities exceeding $n>10^{10}$ \percc
(possibly $n>10^{12}$ \percc, depending on which collisional coefficients are
used) are required to have collisions dominate over radiation. 


% Either
% the higher-excitation lines are too faint {\color{red} do we expect this?}
% or the salt emission is coming from a region of the disk that is shielded
% behind an extinction layer.   Since the salts would likely be incorporated into
% grains {\color{red}can we back this?} if a large population of grains
% were present, it is not clear that such a layer can exist.



% We plot and fit rotational diagrams to measure the abundance and temperature of
% these molecules.  However, the results are ambiguous: for each vibrational
% state with more than one detected transition, we are able to fit an excitation
% temperature, but the abundances of the vibrationally excited states are much
% higher than the ground state.  This distribution implies that the lines are
% radiatively excited.
%
% The excitation temperatures derived from the rotational diagram fits are in the
% range 50-100 K.  The
% radiation temperature within 50 AU of a $10^4$ \lsun star should be $T_R>500$ K,
% so how do we get lower temperatures?
% {\color{red} This is a place for some discussion: is it possible that the gas
% is cooler and the transitions we're seeing are affected by collisions?
% This is the process invoked in \citet{Kaplan2017a}, but it doesn't make sense here:
% it requires the gas to be below the radiation temperature.  Also, the critical density
% of these transitions is likely to be absurdly high, given their A-values in the 0.1s
% range.}



\subsection{Why are the salts present?}

Sodium and potassium are rarely observed in the dense molecular interstellar
medium.  They are usually assumed to be rapidly incorporated into dust grains
after being ejected from dying stars \citep[e.g.][]{Milam2007a}.  Since we
clearly observe NaCl and KCl in the atmosphere of the disk, it is clear that
there is a zone where either dust has not yet formed or
where dust is destroyed and returned to the gas phase.

% I really hope someone takes this too seriously...
Most likely, the lack of salt in the outflow is because it's dissolved into
the \water.  The outflow is a beach.

% If we assume the column density for the v=0 state represents the total
% column of KCl - which is a fairly bad assumption - we can infer the relative abundance.
% We assume the disk has a column density

% {\color{red} What can
% we say about the dust destruction/sublimation zone?  I need to do some disk
% research; there must be a dust sublimation zone at the surfaces of disks in
% addition to the radial one.}

\subsubsection{Highly Speculative discussion}
Why are there salts?  Some possibilities that relate to the formation of the system:

(1) the dust is totally vaporized, so the salts are in the gas phase.  This is
pretty implausible, since it presumably requires temperatures >1000 K,
and the highest excitation line can be excited by 4 \um photons.
Also, if we trust the rotational temperatures, there has to be dust
shielding to keep the salt-emitting layer cool.  If we trust those temperatures,
it is hard to understand why the molecules haven't condensed into grains
already, unless the dust chemical equilibrium timescale is long.  But, because
RCB stars exist, I think the dust formation timescale is very tiny, more like
months or years?

(2) the salts are just forming now as they’re released from the star.  This
doesn’t work at all if the star is accreting, BUT if the collision hypothesis
is correct, we might see tons of stellar interior products mixed into the
disk’s atmosphere.  This scenario requires the adsorption timescale for the
salts to be $>500$ yr, which again seems to contradict the existence of RCB
stars that rapidly form dust at probably comparable or lower densities.
Also, I believe Na, K, and Cl are supernova products, not stellar nucleosynthesis
products, but I am not sure of that!

\bam{(3) Stupid idea: the salts are ejected from the star, do condense out, and then are being ejected again in the outer edges of the disk as there is some sort of turbulent mixing there grinding the dust back up?  If that's the case though, wouldn't we expect to see SiO there as well? hm...}

\subsection{Future prospects}
The detection of these refractory species in the gas phase in the atmosphere
of \sourcei's disk hints that other refractory molecules may be present, 
which would enable their first detection in the ISM (e.g., such rare species as
FeO and FeS), and may enable direct measurments of the metallicity in
star-forming gas.
It remains unclear whether \sourcei is a unique source or is representative
of the class of high-mass protostars with disks; if the latter, these lines
and potentially many others can be used to probe the radiation environment of 
extremely embedded high-mass protostars.

\begin{figure*}[!htp]
\includegraphics[scale=1,width=3.5in]{figures/KCl_rotational_diagrams.pdf}
\includegraphics[scale=1,width=3.5in]{figures/NaCl_rotational_diagrams.pdf}
\caption{Rotational energy diagrams for the KCl and NaCl lines.  While each
vibrational state can internally be explained reasonably well by a single
consistent rotation temperature in the range $T_{rot}\sim50-150$ K, the population
distribution between vibrational excitation states cannot.}
\label{fig:rotationdiagrams}
\end{figure*}

\begin{table*}[htp]
\centering
\caption{NaCl Lines}
\begin{tabular}{ccccccc}
\label{tab:NaCl_salt_lines}
 J$_u$ & J$_l$ & Frequency & Velocity & Width & Amplitude & E$_U$ \\
  &  & $\mathrm{GHz}$ & $\mathrm{km\,s^{-1}}$ & $\mathrm{km\,s^{-1}}$ & $\mathrm{K}$ & $\mathrm{K}$ \\
\hline
&\vspace{-0.75em}\\
\multicolumn{7}{c}{$v = 0$} \\
\vspace{-0.75em}\\
0 & 45 & 44 & 335.05221 & 6.2 (0.4) & 6.4 (0.4) & 12.1 (0.7) & 370.4 \\
 18 & 17 & 229.24720 & 3.3 (0.1) & 4.7 (0.1) & 21.4 (0.4) & 104.6 \\
 7 & 6 & 89.22066 & 2.5 (0.4) & 5.0 (0.4) & 18.2 (1.2) & 17.1 \\
&\vspace{-0.75em}\\
\multicolumn{7}{c}{$v = 2$} \\
\vspace{-0.75em}\\
 31 & 30 & 229.68076 & 4.2 (0.6) & 13.1 (1.0) & 9.6 (0.3) & 962.5 \\
 13 & 12 & 98.70507 & 4.3 (0.9) & 7.2 (1.0) & 13.6 (1.4) & 828.3 \\
 18 & 17 & 230.77811 & 6.9 (0.1) & 6.0 (0.1) & 19.5 (0.4) & 1138.6 \\
&\vspace{-0.75em}\\
\multicolumn{7}{c}{$v = 5$} \\
\vspace{-0.75em}\\
 45 & 44 & 334.29723 & 2.7 (0.3) & 16.6 (0.6) & 42.3 (0.5) & 2332.2 \\
 31 & 30 & 230.72302 & 6.5 (0.4) & 4.3 (0.4) & 5.2 (0.5) & 2139.8 \\
&\vspace{-0.75em}\\
\multicolumn{7}{c}{$v = 6$} \\
\vspace{-0.75em}\\
 30 & 29 & 215.73578 & 7.3 (2.8) & 3.0 (2.8) & 1.9 (1.5) & 2473.0 \\
\hline
\end{tabular}

\par 
\end{table*}

\begin{table*}[htp]
\centering
\caption{Na$^{37}$Cl Lines}
\begin{tabular}{ccccccc}
\label{tab:Na37Cl_salt_lines}
 J$_u$ & J$_l$ & Frequency & Velocity & Width & Amplitude & E$_U$ \\
  &  & $\mathrm{GHz}$ & $\mathrm{km\,s^{-1}}$ & $\mathrm{km\,s^{-1}}$ & $\mathrm{K}$ & $\mathrm{K}$ \\
\hline
&\vspace{-0.75em}\\
\multicolumn{7}{c}{$v = 0$} \\
\vspace{-0.75em}\\
 7 & 6 & 89.22011 & 4.3 (0.4) & 5.0 (0.4) & 18.2 (1.2) & 17.1 \\
 18 & 17 & 229.24605 & 4.8 (0.1) & 4.7 (0.1) & 21.4 (0.4) & 104.6 \\
 17 & 16 & 214.93871 & 4.8 (0.4) & 4.6 (0.4) & 18.7 (1.2) & 607.0 \\
&\vspace{-0.75em}\\
\multicolumn{7}{c}{$v = 1$} \\
\vspace{-0.75em}\\
 8 & 7 & 101.21188 & 4.6 (0.3) & 5.7 (0.3) & 19.7 (0.8) & 536.0 \\
 8 & 7 & 100.46695 & 5.3 (0.4) & 7.9 (0.5) & 15.6 (0.7) & 1045.0 \\
&\vspace{-0.75em}\\
\multicolumn{7}{c}{$v = 2$} \\
\vspace{-0.75em}\\
 7 & 6 & 87.91232 & 4.3 (0.6) & 6.3 (0.6) & 13.1 (1.1) & 1040.1 \\
 7 & 6 & 87.26464 & 2.9 (1.0) & 4.5 (1.0) & 9.0 (1.7) & 1544.3 \\
&\vspace{-0.75em}\\
\multicolumn{7}{c}{$v = 3$} \\
\vspace{-0.75em}\\
 8 & 7 & 99.72675 & 3.7 (0.4) & 5.8 (0.4) & 14.0 (0.8) & 1549.1 \\
 7 & 6 & 86.62109 & -2.1 (1.1) & 8.1 (1.2) & 13.0 (1.4) & 2043.6 \\
&\vspace{-0.75em}\\
\multicolumn{7}{c}{$v = 4$} \\
\vspace{-0.75em}\\
 8 & 7 & 98.99127 & 4.6 (1.1) & 4.2 (1.1) & 8.2 (1.9) & 2048.4 \\
 27 & 26 & 333.45906 & 5.5 (0.5) & 4.8 (0.5) & 10.9 (1.0) & 2251.3 \\
 19 & 18 & 233.16920 & 5.2 (0.5) & 3.9 (0.5) & 7.4 (0.8) & 2633.6 \\
\hline
&\vspace{-0.75em}\\
\multicolumn{7}{c}{$v = 5$} \\
\vspace{-0.75em}\\
 7 & 6 & 85.98167 & 5.2 (1.2) & 4.9 (1.2) & 8.1 (1.7) & 2538.2 \\
\end{tabular}

\par 
\end{table*}

\begin{table*}[htp]
\centering
\caption{KCl Lines}
\begin{tabular}{ccccccc}
\label{tab:KCl_salt_lines}
 J$_u$ & J$_l$ & Frequency & Velocity & Width & Amplitude & E$_U$ \\
  &  & $\mathrm{GHz}$ & $\mathrm{km\,s^{-1}}$ & $\mathrm{km\,s^{-1}}$ & $\mathrm{K}$ & $\mathrm{K}$ \\
\hline
&\vspace{-0.75em}\\
\multicolumn{7}{c}{$v = 0$} \\
\vspace{-0.75em}\\
 28 & 27 & 215.00828 & 5.5 (0.6) & 4.3 (0.6) & 10.4 (1.3) & 149.7 \\
 30 & 29 & 230.32064 & 4.6 (0.2) & 4.8 (0.2) & 14.4 (0.4) & 171.4 \\
 45 & 44 & 344.82061 & 2.5 (1.1) & 4.9 (1.1) & 9.7 (1.8) & 381.2 \\
 13 & 12 & 99.92952 & 4.0 (0.3) & 4.8 (0.3) & 15.3 (0.9) & 33.6 \\
&\vspace{-0.75em}\\
\multicolumn{7}{c}{$v = 1$} \\
\vspace{-0.75em}\\
 44 & 43 & 335.13396 & 3.3 (0.3) & 4.1 (0.3) & 14.1 (0.8) & 761.7 \\
&\vspace{-0.75em}\\
\multicolumn{7}{c}{$v = 2$} \\
\vspace{-0.75em}\\
 13 & 12 & 98.70595 & 2.0 (0.8) & 6.4 (0.8) & 14.2 (1.5) & 828.3 \\
&\vspace{-0.75em}\\
\multicolumn{7}{c}{$v = 3$} \\
\vspace{-0.75em}\\
 13 & 12 & 98.09753 & 6.1 (1.1) & 5.0 (1.1) & 9.2 (1.7) & 1220.6 \\
 29 & 28 & 218.57971 & 4.4 (0.4) & 3.3 (0.4) & 5.0 (0.6) & 1345.1 \\
&\vspace{-0.75em}\\
\multicolumn{7}{c}{$v = 4$} \\
\vspace{-0.75em}\\
 13 & 12 & 97.49133 & 8.0 (2.1) & 7.0 (2.4) & 6.1 (1.5) & 1609.5 \\
 29 & 28 & 217.22891 & -0.3 (0.6) & 3.2 (0.6) & 3.5 (0.6) & 1733.2 \\
&\vspace{-0.75em}\\
\multicolumn{7}{c}{$v = 5$} \\
\vspace{-0.75em}\\
 31 & 30 & 230.72399 & 5.3 (0.4) & 4.3 (0.4) & 5.2 (0.5) & 2139.8 \\
 29 & 28 & 215.88373 & 6.3 (2.1) & 4.3 (2.1) & 3.0 (1.3) & 2118.0 \\
&\vspace{-0.75em}\\
\multicolumn{7}{c}{$v = 6$} \\
\vspace{-0.75em}\\
 47 & 46 & 346.87489 & 0.7 (1.1) & 5.5 (1.2) & 4.4 (0.8) & 2745.3 \\
 29 & 28 & 214.54412 & 6.3 (1.9) & 4.4 (1.9) & 3.4 (1.3) & 2499.6 \\
\hline
\end{tabular}

\par 
\end{table*}

\begin{table*}[htp]
\centering
\caption{K$^{37}$Cl Lines}
\begin{tabular}{ccccccc}
\label{tab:K37Cl_salt_lines}
 J$_u$ & J$_l$ & Frequency & Velocity & Width & Amplitude & E$_U$ \\
  &  & $\mathrm{GHz}$ & $\mathrm{km\,s^{-1}}$ & $\mathrm{km\,s^{-1}}$ & $\mathrm{K}$ & $\mathrm{K}$ \\
\hline
&\vspace{-0.75em}\\
\multicolumn{7}{c}{$v = 0$} \\
\vspace{-0.75em}\\
 45 & 44 & 335.05221 & 6.2 (0.4) & 6.4 (0.4) & 12.1 (0.7) & 370.4 \\
&\vspace{-0.75em}\\
\multicolumn{7}{c}{$v = 1$} \\
\vspace{-0.75em}\\
 12 & 11 & 89.08229 & 5.2 (1.2) & 4.7 (1.2) & 5.5 (1.2) & 421.4 \\
 31 & 30 & 229.81801 & 6.5 (0.3) & 4.8 (0.3) & 7.2 (0.4) & 570.2 \\
\hline
&\vspace{-0.75em}\\
\multicolumn{7}{c}{$v = 6$} \\
\vspace{-0.75em}\\
 30 & 29 & 215.73578 & 7.3 (2.8) & 3.0 (2.8) & 1.9 (1.5) & 2473.0 \\
\end{tabular}

\par 
\end{table*}

\begin{table*}[htp]
\centering
\caption{$^{41}$KCl Lines}
\begin{tabular}{ccccccc}
\label{tab:41KCl_salt_lines}
 J$_u$ & J$_l$ & Frequency & Velocity & Width & Amplitude & E$_U$ \\
  &  & $\mathrm{GHz}$ & $\mathrm{km\,s^{-1}}$ & $\mathrm{km\,s^{-1}}$ & $\mathrm{K}$ & $\mathrm{K}$ \\
\hline
&\vspace{-0.75em}\\
\multicolumn{7}{c}{$v = 0$} \\
\vspace{-0.75em}\\
 29 & 28 & 217.54351 & 5.0 (0.7) & 6.5 (0.7) & 4.5 (0.4) & 156.7 \\
 13 & 12 & 97.62803 & 4.9 (2.5) & 10.0 (3.2) & 6.5 (1.3) & 32.8 \\
&\vspace{-0.75em}\\
\multicolumn{7}{c}{$v = 2$} \\
\vspace{-0.75em}\\
 12 & 11 & 89.03049 & 8.2 (0.0) & 0.2 (0.0) & 0.1 (1.6) & 813.8 \\
\hline
\end{tabular}

\par 
\end{table*}

\begin{table*}[htp]
\centering
\caption{$^{41}$K$^{37}$Cl Lines}
\begin{tabular}{ccccccc}
\label{tab:41K37Cl_salt_lines}
 J$_u$ & J$_l$ & Frequency & Velocity & Width & Amplitude & E$_U$ \\
  &  & $\mathrm{GHz}$ & $\mathrm{km\,s^{-1}}$ & $\mathrm{km\,s^{-1}}$ & $\mathrm{K}$ & $\mathrm{K}$ \\
\hline
\hline
&\vspace{-0.75em}\\
\multicolumn{7}{c}{$v = 7$} \\
\vspace{-0.75em}\\
7 & 48 & 47 & 334.32520 & -9.5 (0.0) & 7.1 (0.3) & 27.8 (0.8) & 3049.3 \\
\end{tabular}

\par 
\end{table*}



\section{Conclusions}

%These lines trace a rare physical phenomenon.  They are therefore an excellent tracer
%of this phenomenon, assuming they are detected

\acknowledgements
The National Radio Astronomy Observatory is a facility of the National Science
Foundation operated under cooperative agreement by Associated Universities,
Inc. The Green Bank Observatory is a facility of the National Science
Foundation operated under cooperative agreement by Associated Universities,
Inc. Support for B.A.M. was provided by NASA through Hubble Fellowship grant
\#HST-HF2-51396 awarded by the Space Telescope Science Institute, which is
operated by the Association of Universities for Research in Astronomy, Inc.,
for NASA, under contract NAS5-26555. 
This paper makes use of the following ALMA data: ADS/JAO.ALMA\#2016.1.00165.S
ALMA is a partnership of ESO (representing its member states), NSF (USA) and
NINS (Japan), together with NRC (Canada), MOST and ASIAA (Taiwan), and KASI
(Republic of Korea), in cooperation with the Republic of Chile. The Joint ALMA
Observatory is operated by ESO, AUI/NRAO and NAOJ.

\input{solobib}

\end{document}
