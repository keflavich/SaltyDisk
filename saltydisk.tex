
\documentclass[twocolumn]{aastex62}
\input{preface}
\newcommand{\sourcei}{SrcI\xspace}
\newcommand{\sourcen}{SrcN\xspace}
\newcommand{\sourcex}{SrcX\xspace}
\newcommand{\bam}[1]{\textcolor{green!65!black}{\textbf{[BAM: #1]}}}
\newcommand{\rlp}[1]{\textcolor{blue!65!black}{\textbf{[RLP: #1]}}}
\newcommand{\ag}[1]{\textcolor{red!65!black}{\textbf{[AG: #1]}}}
\begin{document}
\input{authors}

\title{Orion \sourcei's disk is salty}
\begin{abstract}
    We report the detection of NaCl and KCl and their $^{37}$Cl and $^{41}$K
    isotopologues toward the disk around Orion \sourcei.  About 60 transitions
    of these molecules were identified.
    This is the first detection of these molecules in the interstellar
    medium not associated with the ejecta of evolved stars.  It is also
    the first ever detection of the vibrationally excited states of these
    lines in the ISM above $v=1$, with firm detections up to $v=6$.
    The salt emission traces the region just above the continuum disk,
    possibly forming the base of the outflow.  We examine several possible
    explanations of the observed high excitation lines, concluding that the
    vibrational states are most likely to be radiatively excited via
    rovibrational transitions in the 25-35 \um (NaCl) and 35-45 \um (KCl)
    range.  These molecules may provide a unique toolkit for measuring
    the radiation environment around accreting young stars.
\end{abstract}

\section{Introduction}
The disk around Orion Source I (\sourcei) has been the subject of intense
study, as it is the closest disk around a `high-mass' ($M>8$ \msun) star
\citep{Hirota2014a,Plambeck2016a,Ginsburg2018b}.  It is also in many ways
a unique source, being the only known protostellar object with SiO and \water
masers in an outflow \citep{Goddi2009a,Goddi2010a,Greenhill2013a}.
In \citet{Ginsburg2018b}, a subset of the authors reported 0.03$\arcsec$ ALMA observations of \sourcei
that allowed us to measure the outer rotation curve of the
5(5,0)-6(4,3) \water line in the disk and along the base
of the outflow; these measurements showed that the interior mass was
$15 \pm 2$ \msun.  In the same data, we found additional emission lines that
closely traced the disk and its Keplerian rotation
curve, but were unable to identify the molecular species responsible
for these lines.
We now identify them as transitions of NaCl and KCl
and their isotopologues, many in excited vibrational levels.

Molecules consisting of alkali metals and halogens have only rarely been
detected in space (McGuire 2018, ApJS, in press).
%\citep[e.g., see the review
%by][]{XXX} \bam{This reference should be `McGuire 2018 ApJs in press' for now.}.
Because of their rarity, their use as a diagnostic
tool to measure metallicity and local physical conditions has been limited.
%\rlp{I would consider adding a sentence about how we will use these lines
%to study conditions in Source I, and move the rest of this material to the discussion.}
However, some of these molecules, such as KCl and NaCl, have a rich spectrum
with dozens of lines arising in a single observing band with modern
instrumentation, so they have great potential to probe either local gas properties
or the radiation field when they are detected.


KCl and NaCl have so far been observed toward evolved stars including the carbon star IRC
+10216 \citep{Cernicharo1987a}, oxygen-rich evolved stars IK Tauri and VY Canis
Majoris \citep{Milam2007a}, the post-AGB star envelope CRL 2688
\citep{Highberger2003a}, and the AGB star wind around OH 231+4.2
\citep{Sanchez-Contreras2018a}.  In these sources, they exist only in a limited
range of radii from the central star.  The limited environments in which these
molecules have been detected suggests that they only persist in the gas phase
for a brief period when being ejected from dying stars before being
incorporated into dust grains.
This transient gas-phase abundance is analogous to that which makes SiO an
excellent tracer of recent shock events: Si is sputtered from grains into the
gas phase where it reacts with oxygen to form SiO.  Further reactions rapidly
form SiO$_2$, depleting the gas-phase SiO abundance, and making that molecule
an indicator of a recent shock \citep{Schilke1997a}.  Thus, if the production
pathways, and role of the physical environment on those pathways, can be
constrained for NaCl and KCl, the presence of these molecules has the potential
to be a powerful tracer of physical conditions and history in a source. 

% \bam{Did we check w/ Holger here to see where he thought
% the v=1 for NaCl was seen?  I still can't find a paper w/ it.  It's also
% probably not correct to cite my paper here - I haven't (yet) compiled
% vibrational information...} \ag{I haven't e-mailed him; you want to do that?
% I was just citing your paper since the information came from your brain...
% we ought to cite you somewhere, if not here}

% \rlp{I commented out this paragraph because some of it repeats what is
% in the first paragraph, and the rest can be covered in the discussion.}
% The environments around forming stars, however, are also known to harbor
% molecules that are often rapidly depleted from the gas-phase after ejection.
% Disks around forming high-mass stars have been resolved only a handful of
% times.  The disk around Orion Source I (\sourcei) has frequently been observed
% \citep{Hirota2014a,Plambeck2016a,Ginsburg2018b}, but most molecular species
% detected trace either an outflow (e.g., SiO) or some combination of the outflow
% and the disk (e.g., \water).  In HH80/81, a $\sim4$ \msun disk around a
% $\sim10-20$ \msun star, several lines of SO$_2$ are associated with the disk
% \citep{Girart2017a}.  Beyond these two sources, only disordered structures or
% disk-envelope blends have been observed.

% \bam{In HH80/81, or do you mean beyond
% this pair only disordered blends have been observed period, anywhere?}
% \ag{AFAIK, except for the HH80/81 and \sourcei disks, all disks around massive
% stars are ambiguous / not strictly disks.}

% In \citet{Ginsburg2018b}, we used lines of \water and one or
% more unknown species to measure the rotation curve of the disk around \sourcei.
% We have now identified these unknown lines as transitions of NaCl and KCl.

\section{Observations and Analysis}

% rlp: I have moved pieces of this paragraph - but not all of it - to later in this section
% The pure-rotational spectra of KCl and its rare isotopologues was reported by
% \citet{Caris2004a} up to 930~GHz.  Most isotopologues were measured up to the
% seventh excited vibrational state.  \citet{Caris2004a} also refine predictions
% for NaCl based on prior laboratory work (again to 930~GHz) reported by
% \citet{Caris2002a}.  The results of these works were then subsequently used by
% \citet{Barton2014a} to generate high-temperature line lists, primarily for use
% in exoplanet atmospheres, and includes over 10$^5$ transitions for each
% molecular species and isotopologue with levels up to $E_U\sim5\ee{4}$ K. The
% primary result for the purposes of this work is the prediction of frequencies
% in vibrational states higher than those measured in the laboratory. To our
% knowledge, prior to this work, no astronomical detections of vibrationally
% excited transitions of either of these species, or their isotopologues, have
% been reported.


The observations presented here are described in \citet{Ginsburg2018b} as part
of ALMA project 2016.1.00165.S.  We use the robust 0.5 weighted spectral cubes
from all three bands (B3 3 mm, B6 1 mm, B7 0.85 mm) for our spectroscopic analysis.

Appendix D of that paper describes the spectral extraction method,
which we summarize here.  We used the U232.511 line (which we now identify as
NaCl v=1 J=18-17) to find the velocity centroid for each (spatial) pixel
across the \sourcei disk.  We shifted the spectrum along
each such pixel to 0 \kms, then averaged the shifted spectra over the region with
significant emission in the NaCl line.  
The
averaging area is approximately the extent of the continuum disk, 0.03 square
arcseconds, or about 4, 20, and 40  beam areas, resulting
in an improvement in the signal-to-noise of about $2\times$, $4\times$, and $6\times$
respectively in Band 3, 6, and 7.
This stacked spectrum, shown in Figures \ref{fig:spectrab3}-\ref{fig:spectrab7},
% is the average spectrum across the disk identified in \citet{Ginsburg2018b}: it
traces material just above and below the optically thick continuum disk
(Figure \ref{fig:spatial}), and for the lines discussed here, includes no
emission associated with the outflow.  
% OK, I'm putting my preferred sentence here for now. It could be moved
% to a footnote or the figure caption if desired. - Dick
% AG: I don't like this because it sounds like the lines are narrower than we
% observe, which isn't right: they are narrower than what you get by averaging
% over the observations, but they are wider than you get by pointing a single
% beam at the disk as is.
Because this procedure removes the rotational velocity field of the disk,
lines in the stacked spectrum are narrower than one would observe in
a lower spatial resolution observation of the disk.  One may think of the
stacked spectrum as the output of a matched filter that is optimized for
detection of disk emission.

% The stacked spectrum is equivalent to that
% obtained with a matched filter using a narrow (delta-function) filter at each
% position tuned to the appropriate centroid velocity.

% We have clearly identified the carrier of the majority of unidentified lines in
% the \citet{Ginsburg2018b} spectra (only 2-3 lines are unidentified in the B6/B7
% spectra, though there are several in B3 we have not identified).

In the appendix of \citet{Ginsburg2018b}, we listed over 20 unidentified 
lines in the B6 ALMA spectra, and commented that ``there is no
consistent pattern to the detected lines and no individual species can explain
more than a few of the observed lines.''  This statement was incorrect, as
there are obvious carriers for the majority of the unidentified lines that we
had simply overlooked: NaCl, KCl, and their isotopologues.  The
detected lines have amplitudes in the range 0.5-3 mJy \perbeam, corresponding
to brightness temperatures 5-20 K.  

In Figures \ref{fig:spectrab3}-\ref{fig:spectrab7}, we have labeled all of the
detections and marginal detections of salt lines, as well as the most prominent
outflow (e.g., SiO and H$_2$O) lines.  Only 2-3 emission lines are now
unidentified in the B6 and B7 spectra, though there are over a dozen in B3 that
we have not identified. 

We also tentatively identify broad lines at 229.7 GHz and 334.4 GHz as AlO.  If
the identification is correct, these lines are broad because of the rich hyperfine
spectrum of AlO.

% I count 15 unidentified emission lines in B3; I added "emission" above 
% because there are lots of unidentified absorption lines from foreground gas. - Dick
%
%We additionally labeled
%\emph{some} of the lines that were not detected because of severe confusion

\begin{figure*}[!htp]
\includegraphics[scale=1,width=5.5in]{figures/lines_labeled_OrionSourceI_B3_spw0_robust0.5.pdf}
\includegraphics[scale=1,width=5.5in]{figures/lines_labeled_OrionSourceI_B3_spw1_robust0.5.pdf}
\includegraphics[scale=1,width=5.5in]{figures/lines_labeled_OrionSourceI_B3_spw2_robust0.5.pdf}
\includegraphics[scale=1,width=5.5in]{figures/lines_labeled_OrionSourceI_B3_spw3_robust0.5.pdf}
\caption{Stacked spectra  B3}
\label{fig:spectrab3}
\end{figure*}
\begin{figure*}[!htp]
\includegraphics[scale=1,width=5.5in]{figures/lines_labeled_OrionSourceI_B6_spw0_robust0.5.pdf}
\includegraphics[scale=1,width=5.5in]{figures/lines_labeled_OrionSourceI_B6_spw1_robust0.5.pdf}
\includegraphics[scale=1,width=5.5in]{figures/lines_labeled_OrionSourceI_B6_spw2_robust0.5.pdf}
\includegraphics[scale=1,width=5.5in]{figures/lines_labeled_OrionSourceI_B6_spw3_robust0.5.pdf}
\caption{Stacked spectra B6}
\label{fig:spectrab6}
\end{figure*}
\begin{figure*}[!htp]
\includegraphics[scale=1,width=5.5in]{figures/lines_labeled_OrionSourceI_B7.lb_spw0_robust0.5.pdf}
\includegraphics[scale=1,width=5.5in]{figures/lines_labeled_OrionSourceI_B7.lb_spw1_robust0.5.pdf}
\includegraphics[scale=1,width=5.5in]{figures/lines_labeled_OrionSourceI_B7.lb_spw2_robust0.5.pdf}
\includegraphics[scale=1,width=5.5in]{figures/lines_labeled_OrionSourceI_B7.lb_spw3_robust0.5.pdf}
\caption{Stacked spectra B7}
\label{fig:spectrab7}
\end{figure*}


In Tables \ref{tab:all_detections_B3}-\ref{tab:all_detections_B7}, we list all
of the salt lines in vibrational energy levels up to v=8 that lie within our
observed bands and whether or not they were detected.  We employ the following
flag scheme: `d' for detection, `n' for nondetection, `q' for questionable
(e.g., low signal-to-noise, but possibly detected, or in regions where the
noise is not Gaussian and may be affected by other lines).  We additionally
include a flag `c' for `confused', which we add to the flag string if the line
is blended with another brighter line of a different species.  There are robust
detections of about 60 lines, up to the v=6 vibrational level.

The majority of the NaCl and KCl lines that we detected are in excited
vibrational states.  Although transitions in higher vibrational states were
measured, and some reported, in \citet{Caris2004a}, these did not provide the
complete catalogs needed for this analysis, nor to our knowledge are these
catalogues available in any of the standard online databases \citep[i.e., CDMS,
SLAIM, JPL;][]{Muller2005a,Lovas2005a,Pickett1998a}.  We obtained the KCl
frequencies for these lines from the catalogs of \citet{Barton2014a}, which was
designed primarily for use in exoplanet atmospheres and includes over 10$^5$
transitions for each molecular species and isotopologue with levels up to
$E_U\sim5\ee{4}$ K, and the NaCl frequencies from \citet{Cabezas2016a}, who
provided more accurate potential energy and dipole moment functions. The
primary result of these catalogs for the purposes of this work is the
prediction of frequencies in vibrational states higher than those measured in
the laboratory. 
% \rlp{Would it be fair just to say that we used Cabezas for the NaCl
% frequencies and Barton for the KCl frequencies?}

% The results of these works were then subsequently used by
% \citet{Barton2014a} to generate high-temperature line lists, primarily for use
% in exoplanet atmospheres, and includes over 10$^5$ transitions for each
% molecular species and isotopologue with levels up to $E_U\sim5\ee{4}$ K. The
% primary result for the purposes of this work is the prediction of frequencies
% in vibrational states higher than those measured in the laboratory. 

% We therefore used the predictions from the \citet{Barton2014a} catalog
% to identify lines with vibrational states $v>2$ for KCl and $v>4$ for NaCl.  
% \rlp{There's a paper by Cabezas et al, ApJ 2016 that claims to derive highly
% accurate frequencies, A coefficients, etc for NaCl up to J=150 and v=8; how
% does that fit into this story??}
% \ag{Added a reference.  Mostly, they fit the CDMS/SLAIM catalogs.  I've also switched
% to using the Cabezas values in the tables for NaCl.}

In using these catalogs, we found a systematic discrepancy between the rest
frequencies reported by \citet{Barton2014a}, and those given by
\citet{Caris2004a} and \citet{Caris2002a} and the catalogs generated from them:
the KCl lines appear to be systematically offset by $\sim$15-20 \kms in the
Barton catalog.  The Barton data are also discrepant with the more recent
\citet{Cabezas2016a} work that cataloged only NaCl lines. The offset is a
function of the vibrational and rotational energy states, indicating that it
resulted from an incorrect rotational and/or distortion constant.  We correct
for this error by performing a bilinear fit in frequency as a function of $v_u$
and $J_u$.  In the transitions available in both catalogs the resulting
frequency differences are generally $<0.1$ \kms, which is more than sufficient
for a reliable match to our observational spectra.  We then applied these
fitted models to the higher-$v$ states tabulated by Barton to get corrected
rest frequencies for KCl, while for NaCl we used the \citet{Cabezas2016a}
values.  We note that the values printed in \citet{Barton2014a} differ from
those in the digital catalogs hosted on \url{exomol.com}, so we suspect the
error was merely a transcription error of some sort.
%A more detailed re-fitting and prediction of the frequencies is beyond the
%scope of this \emph{Letter}, but is certainly warranted for follow-up efforts.

% \textbf{\color{red} At the moment, all of the line frequencies are shifted
% to higher frequencies by 17 \kms (KCl) and 3 \kms (NaCl) because of a discrepancy
% between the \citep{Barton2014a} catalog line frequencies and those of CDMS,
% JPL, and SLAIM.  The latter three, catalogued in Splatalogue, agree, and they
% agree better with my measurements.}

\begin{table*}[htp]
\centering
\caption{All detected lines in Band 3}
\begin{tabular}{ccccccc}
\label{tab:all_detections_B3}
Species & v & J$_u$ & J$_l$ & E$_U$ & Frequency & Flag \\
\hline
$^{41}$K$^{35}$Cl & 8 & 12 & 11 & 3092.4 & 85.80766 & -n \\
$^{23}$Na$^{35}$Cl & 8 & 7 & 6 & 4031.9 & 85.86278 & -n \\
$^{39}$K$^{37}$Cl & 7 & 12 & 11 & 2713.3 & 85.87379 & -n \\
$^{41}$K$^{37}$Cl & 3 & 12 & 11 & 1183.9 & 85.93776 & -n \\
$^{23}$Na$^{37}$Cl & 5 & 7 & 6 & 2536.0 & 85.97809 & -d \\
$^{41}$K$^{35}$Cl & 7 & 12 & 11 & 2720.7 & 86.33986 & cn \\
$^{39}$K$^{37}$Cl & 6 & 12 & 11 & 2339.4 & 86.40364 & -n \\
$^{41}$K$^{37}$Cl & 2 & 12 & 11 & 801.6 & 86.45668 & -n \\
$^{23}$Na$^{35}$Cl & 7 & 7 & 6 & 3547.0 & 86.51213 & -q \\
$^{23}$Na$^{37}$Cl & 4 & 7 & 6 & 2041.9 & 86.61903 & -d \\
$^{39}$K$^{35}$Cl & 10 & 12 & 11 & 3869.7 & 86.68041 & -n \\
$^{41}$K$^{35}$Cl & 6 & 12 & 11 & 2345.7 & 86.87413 & -n \\
$^{39}$K$^{37}$Cl & 5 & 12 & 11 & 1962.2 & 86.93553 & -n \\
$^{41}$K$^{37}$Cl & 1 & 12 & 11 & 416.1 & 86.97743 & -q \\
$^{23}$Na$^{35}$Cl & 6 & 7 & 6 & 3057.4 & 87.16544 & -d \\
$^{39}$K$^{35}$Cl & 9 & 12 & 11 & 3500.3 & 87.22792 & -n \\
$^{23}$Na$^{37}$Cl & 3 & 7 & 6 & 1543.0 & 87.26372 & -d \\
$^{41}$K$^{35}$Cl & 5 & 12 & 11 & 1967.6 & 87.41041 & -n \\
$^{39}$K$^{37}$Cl & 4 & 12 & 11 & 1581.9 & 87.46938 & -n \\
$^{41}$K$^{37}$Cl & 0 & 12 & 11 & 27.3 & 87.50010 & -q \\
$^{39}$K$^{35}$Cl & 8 & 12 & 11 & 3127.7 & 87.77742 & -n \\
$^{23}$Na$^{35}$Cl & 5 & 7 & 6 & 2563.0 & 87.82283 & -d \\
$^{23}$Na$^{37}$Cl & 2 & 7 & 6 & 1039.3 & 87.91214 & -d \\
$^{41}$K$^{35}$Cl & 4 & 12 & 11 & 1586.3 & 87.94866 & -n \\
$^{39}$K$^{37}$Cl & 3 & 12 & 11 & 1198.3 & 88.00523 & -d \\
$^{39}$K$^{35}$Cl & 7 & 12 & 11 & 2751.8 & 88.32895 & -q \\
$^{23}$Na$^{35}$Cl & 4 & 7 & 6 & 2063.8 & 88.48417 & -d \\
$^{41}$K$^{35}$Cl & 3 & 12 & 11 & 1201.7 & 88.48896 & cq \\
$^{39}$K$^{37}$Cl & 2 & 12 & 11 & 811.5 & 88.54307 & -d \\
$^{23}$Na$^{37}$Cl & 1 & 7 & 6 & 530.7 & 88.56426 & -d \\
$^{39}$K$^{35}$Cl & 6 & 12 & 11 & 2372.8 & 88.88253 & -n \\
$^{41}$K$^{35}$Cl & 2 & 12 & 11 & 813.8 & 89.03129 & -d \\
$^{39}$K$^{37}$Cl & 1 & 12 & 11 & 421.4 & 89.08292 & -d \\
$^{23}$Na$^{35}$Cl & 3 & 7 & 6 & 1559.6 & 89.14960 & -d \\
$^{41}$K$^{37}$Cl & 10 & 13 & 12 & 3776.8 & 89.21871 & -n \\
$^{23}$Na$^{37}$Cl & 0 & 7 & 6 & 17.1 & 89.22014 & -d \\
$^{39}$K$^{35}$Cl & 4 & 13 & 12 & 1609.5 & 97.49133 & -d \\
$^{23}$Na$^{37}$Cl & 6 & 8 & 7 & 3030.0 & 97.52883 & -q \\
$^{41}$K$^{35}$Cl & 0 & 13 & 12 & 32.8 & 97.62809 & -d \\
$^{41}$K$^{37}$Cl & 7 & 14 & 13 & 2690.7 & 97.85364 & -n \\
$^{39}$K$^{35}$Cl & 3 & 13 & 12 & 1220.6 & 98.09753 & -d \\
$^{23}$Na$^{35}$Cl & 8 & 8 & 7 & 4036.6 & 98.12538 & -q \\
$^{23}$Na$^{37}$Cl & 5 & 8 & 7 & 2540.7 & 98.25682 & -q \\
$^{39}$K$^{37}$Cl & 10 & 14 & 13 & 3825.5 & 98.33904 & -n \\
$^{41}$K$^{37}$Cl & 6 & 14 & 13 & 2321.0 & 98.44993 & -n \\
$^{39}$K$^{35}$Cl & 2 & 13 & 12 & 828.3 & 98.70595 & -d \\
$^{23}$Na$^{35}$Cl & 7 & 8 & 7 & 3551.7 & 98.86719 & -q \\
$^{41}$K$^{35}$Cl & 10 & 14 & 13 & 3835.7 & 98.86759 & cn \\
$^{39}$K$^{37}$Cl & 9 & 14 & 13 & 3461.0 & 98.94969 & -n \\
$^{23}$Na$^{37}$Cl & 4 & 8 & 7 & 2046.6 & 98.98914 & -d \\
$^{41}$K$^{37}$Cl & 5 & 14 & 13 & 1948.2 & 99.04845 & -n \\
$^{39}$K$^{35}$Cl & 1 & 13 & 12 & 432.6 & 99.31663 & -d \\
$^{41}$K$^{35}$Cl & 9 & 14 & 13 & 3470.3 & 99.48329 & -n \\
$^{39}$K$^{37}$Cl & 8 & 14 & 13 & 3093.3 & 99.56276 & -n \\
$^{23}$Na$^{35}$Cl & 6 & 8 & 7 & 3062.2 & 99.61369 & -d \\
$^{41}$K$^{37}$Cl & 4 & 14 & 13 & 1572.3 & 99.64924 & -n \\
$^{23}$Na$^{37}$Cl & 3 & 8 & 7 & 1547.8 & 99.72579 & -d \\
$^{39}$K$^{35}$Cl & 0 & 13 & 12 & 33.6 & 99.92952 & -d \\
$^{41}$K$^{35}$Cl & 8 & 14 & 13 & 3101.7 & 100.10138 & -n \\
$^{39}$K$^{37}$Cl & 7 & 14 & 13 & 2722.6 & 100.17822 & -n \\
$^{41}$K$^{37}$Cl & 3 & 14 & 13 & 1193.2 & 100.25229 & -n \\
$^{23}$Na$^{35}$Cl & 5 & 8 & 7 & 2567.8 & 100.36474 & -d \\
$^{23}$Na$^{37}$Cl & 2 & 8 & 7 & 1044.1 & 100.46673 & -d \\
$^{41}$K$^{35}$Cl & 7 & 14 & 13 & 2730.0 & 100.72190 & -n \\
$^{39}$K$^{37}$Cl & 6 & 14 & 13 & 2348.7 & 100.79603 & -q \\
$^{41}$K$^{37}$Cl & 2 & 14 & 13 & 810.9 & 100.85758 & -n \\
$^{23}$Na$^{35}$Cl & 4 & 8 & 7 & 2068.6 & 101.12050 & -d \\
$^{39}$K$^{35}$Cl & 10 & 14 & 13 & 3879.0 & 101.12088 & -n \\
$^{23}$Na$^{37}$Cl & 1 & 8 & 7 & 535.5 & 101.21200 & -d \\
\hline
\end{tabular}

\par 
\end{table*}

\begin{table*}[htp]
\centering
\caption{All detected lines in Band 6}
\begin{tabular}{ccccccc}
\label{tab:all_detections_B6}
Species & v & J$_u$ & J$_l$ & E$_U$ & Frequency & Flag \\
\hline
$^{39}$K$^{37}$Cl & 7 & 30 & 29 & 2846.1 & 214.41527 & -n \\
$^{23}$Na$^{37}$Cl & 9 & 18 & 17 & 4547.4 & 214.43575 & cn \\
$^{39}$K$^{35}$Cl & 6 & 29 & 28 & 2499.6 & 214.54412 & -d \\
$^{41}$K$^{37}$Cl & 3 & 30 & 29 & 1316.8 & 214.57918 & -n \\
$^{23}$Na$^{35}$Cl & 4 & 17 & 16 & 2139.6 & 214.74036 & cn \\
$^{41}$K$^{35}$Cl & 2 & 29 & 28 & 940.8 & 214.90740 & cq \\
$^{23}$Na$^{37}$Cl & 1 & 17 & 16 & 606.5 & 214.93872 & -d \\
$^{39}$K$^{35}$Cl & 0 & 28 & 27 & 149.7 & 215.00828 & -d \\
$^{39}$K$^{37}$Cl & 1 & 29 & 28 & 548.5 & 215.03464 & cq \\
$^{41}$K$^{37}$Cl & 8 & 31 & 30 & 3187.6 & 215.09368 & cn \\
$^{41}$K$^{35}$Cl & 7 & 30 & 29 & 2854.2 & 215.57755 & cn \\
$^{39}$K$^{37}$Cl & 6 & 30 & 29 & 2473.0 & 215.73679 & -d \\
$^{41}$K$^{37}$Cl & 2 & 30 & 29 & 935.3 & 215.87559 & -n \\
$^{39}$K$^{35}$Cl & 5 & 29 & 28 & 2118.0 & 215.88373 & -d \\
$^{23}$Na$^{37}$Cl & 8 & 18 & 17 & 4072.5 & 216.04057 & -n \\
$^{39}$K$^{37}$Cl & 5 & 30 & 29 & 2096.7 & 217.06391 & -q \\
$^{41}$K$^{37}$Cl & 1 & 30 & 29 & 550.6 & 217.17723 & -q \\
$^{39}$K$^{35}$Cl & 4 & 29 & 28 & 1733.2 & 217.22891 & cd \\
$^{23}$Na$^{35}$Cl & 10 & 18 & 17 & 5070.5 & 217.31725 & cn \\
$^{39}$K$^{37}$Cl & 10 & 31 & 30 & 3957.2 & 217.47744 & -n \\
$^{41}$K$^{35}$Cl & 0 & 29 & 28 & 156.7 & 217.54317 & -d \\
$^{23}$Na$^{37}$Cl & 7 & 18 & 17 & 3593.1 & 217.65560 & -q \\
$^{41}$K$^{37}$Cl & 6 & 31 & 30 & 2452.9 & 217.72366 & -n \\
$^{39}$K$^{35}$Cl & 9 & 30 & 29 & 3635.2 & 217.79704 & cn \\
$^{23}$Na$^{35}$Cl & 2 & 17 & 16 & 1127.5 & 217.97998 & -d \\
$^{41}$K$^{35}$Cl & 5 & 30 & 29 & 2102.8 & 218.24805 & -n \\
$^{39}$K$^{37}$Cl & 4 & 30 & 29 & 1717.2 & 218.39657 & cn \\
$^{41}$K$^{37}$Cl & 0 & 30 & 29 & 162.6 & 218.48414 & cn \\
$^{39}$K$^{35}$Cl & 3 & 29 & 28 & 1345.1 & 218.57971 & -d \\
$^{41}$K$^{35}$Cl & 10 & 31 & 30 & 3968.1 & 218.64471 & -n \\
$^{39}$K$^{37}$Cl & 9 & 31 & 30 & 3593.5 & 218.82568 & -q \\
$^{23}$Na$^{37}$Cl & 0 & 18 & 17 & 104.6 & 229.24601 & -d \\
$^{39}$K$^{35}$Cl & 6 & 31 & 30 & 2521.2 & 229.29217 & cd \\
$^{23}$Na$^{35}$Cl & 10 & 19 & 18 & 5081.5 & 229.36540 & cn \\
$^{41}$K$^{35}$Cl & 2 & 31 & 30 & 962.5 & 229.68227 & cd \\
$^{23}$Na$^{37}$Cl & 7 & 19 & 18 & 3604.1 & 229.72316 & cn \\
$^{39}$K$^{37}$Cl & 1 & 31 & 30 & 570.2 & 229.81880 & -d \\
$^{41}$K$^{35}$Cl & 7 & 32 & 31 & 2875.9 & 229.90046 & -q \\
$^{39}$K$^{37}$Cl & 6 & 32 & 31 & 2494.7 & 230.07072 & -d \\
$^{41}$K$^{37}$Cl & 2 & 32 & 31 & 957.1 & 230.22070 & -q \\
$^{41}$K$^{37}$Cl & 7 & 33 & 32 & 2843.6 & 230.31852 & cn \\
$^{39}$K$^{35}$Cl & 0 & 30 & 29 & 171.4 & 230.32064 & -d \\
$^{39}$K$^{35}$Cl & 5 & 31 & 30 & 2139.8 & 230.72399 & -d \\
$^{23}$Na$^{35}$Cl & 2 & 18 & 17 & 1138.6 & 230.77883 & -d \\
$^{39}$K$^{35}$Cl & 10 & 32 & 31 & 4025.6 & 230.81309 & -n \\
$^{39}$K$^{35}$Cl & 4 & 31 & 30 & 1755.1 & 232.16185 & -d \\
$^{39}$K$^{35}$Cl & 9 & 32 & 31 & 3657.1 & 232.26613 & cn \\
$^{41}$K$^{35}$Cl & 0 & 31 & 30 & 178.7 & 232.49980 & cn \\
$^{23}$Na$^{35}$Cl & 1 & 18 & 17 & 625.2 & 232.50998 & -d \\
$^{41}$K$^{35}$Cl & 10 & 33 & 32 & 3990.1 & 232.69933 & cn \\
$^{41}$K$^{35}$Cl & 5 & 32 & 31 & 2124.8 & 232.74872 & -q \\
$^{23}$Na$^{35}$Cl & 8 & 19 & 18 & 4127.2 & 232.84920 & -q \\
$^{39}$K$^{37}$Cl & 9 & 33 & 32 & 3615.5 & 232.89230 & -n \\
$^{39}$K$^{37}$Cl & 4 & 32 & 31 & 1739.2 & 232.90755 & -d \\
$^{41}$K$^{37}$Cl & 10 & 34 & 33 & 3942.7 & 232.97243 & cn \\
$^{41}$K$^{37}$Cl & 0 & 32 & 31 & 184.6 & 233.00319 & cn \\
$^{41}$K$^{37}$Cl & 5 & 33 & 32 & 2102.9 & 233.12954 & cn \\
$^{23}$Na$^{37}$Cl & 5 & 19 & 18 & 2631.4 & 233.16458 & cd \\
$^{39}$K$^{35}$Cl & 3 & 31 & 30 & 1367.1 & 233.60570 & -d \\
$^{39}$K$^{35}$Cl & 8 & 32 & 31 & 3285.5 & 233.72540 & -n \\
\hline
\end{tabular}

\par 
\end{table*}

\begin{table*}[htp]
\centering
\caption{All detected lines in Band 7}
\begin{tabular}{ccccccc}
\label{tab:all_detections_B7}
Species & v & J$_u$ & J$_l$ & E$_U$ & Frequency & Flag \\
 &  &  &  & $\mathrm{K}$ & $\mathrm{GHz}$ &  \\
\hline
$^{39}$K$^{37}$Cl & 5 & 46 & 45 & 2310.3 & 332.14133 & cn \\
$^{41}$K$^{37}$Cl & 1 & 46 & 45 & 764.4 & 332.34799 & -n \\
$^{41}$K$^{35}$Cl & 2 & 45 & 44 & 1154.0 & 332.81245 & -n \\
$^{23}$Na$^{35}$Cl & 2 & 26 & 25 & 1249.3 & 333.00524 & -d \\
$^{39}$K$^{37}$Cl & 1 & 45 & 44 & 761.8 & 333.01818 & cn \\
$^{39}$K$^{35}$Cl & 2 & 44 & 43 & 1155.3 & 333.06642 & -d \\
$^{41}$K$^{37}$Cl & 4 & 47 & 46 & 1921.1 & 333.42706 & -n \\
$^{23}$Na$^{37}$Cl & 4 & 27 & 26 & 2249.5 & 333.45280 & -d \\
$^{41}$K$^{35}$Cl & 5 & 46 & 45 & 2317.6 & 333.94969 & -n \\
$^{39}$K$^{37}$Cl & 4 & 46 & 45 & 1932.1 & 334.18411 & -n \\
$^{39}$K$^{35}$Cl & 5 & 45 & 44 & 2332.2 & 334.29723 & -d \\
$^{41}$K$^{37}$Cl & 0 & 46 & 45 & 377.7 & 334.35365 & cn \\
$^{41}$K$^{35}$Cl & 1 & 45 & 44 & 764.9 & 334.85440 & cq \\
$^{39}$K$^{37}$Cl & 0 & 45 & 44 & 370.4 & 335.05221 & -d \\
$^{39}$K$^{35}$Cl & 1 & 44 & 43 & 761.7 & 335.13388 & -d \\
$^{41}$K$^{37}$Cl & 3 & 47 & 46 & 1544.2 & 335.44973 & -n \\
$^{23}$Na$^{35}$Cl & 1 & 26 & 25 & 736.8 & 335.50625 & -d \\
$^{41}$K$^{35}$Cl & 0 & 46 & 45 & 389.0 & 344.34007 & cn \\
$^{41}$K$^{37}$Cl & 2 & 48 & 47 & 1180.5 & 344.60831 & cq \\
$^{39}$K$^{35}$Cl & 0 & 45 & 44 & 381.2 & 344.82231 & -d \\
$^{41}$K$^{35}$Cl & 3 & 47 & 46 & 1572.6 & 345.37333 & cn \\
$^{41}$K$^{37}$Cl & 5 & 49 & 48 & 2327.8 & 345.40571 & cn \\
$^{39}$K$^{37}$Cl & 2 & 47 & 46 & 1182.6 & 345.59753 & cn \\
$^{23}$Na$^{37}$Cl & 4 & 28 & 27 & 2266.1 & 345.74820 & cn \\
$^{23}$Na$^{35}$Cl & 2 & 27 & 26 & 1265.9 & 345.75983 & cd \\
$^{39}$K$^{35}$Cl & 3 & 46 & 45 & 1578.4 & 345.94612 & cn \\
$^{39}$K$^{37}$Cl & 5 & 48 & 47 & 2343.3 & 346.47135 & -n \\
$^{41}$K$^{37}$Cl & 1 & 48 & 47 & 797.3 & 346.69213 & cn \\
$^{41}$K$^{35}$Cl & 2 & 47 & 46 & 1187.0 & 347.49658 & -q \\
$^{41}$K$^{37}$Cl & 4 & 49 & 48 & 1954.1 & 347.50516 & -n \\
$^{39}$K$^{37}$Cl & 1 & 47 & 46 & 794.8 & 347.71265 & -d \\
\hline
\end{tabular}

\par 
\end{table*}


%The low observed brightness temperatures suggest that the emission from
%these lines is optically thin.

We have fit Gaussian line profiles to each of the transitions we flagged as
`detected' in Tables \ref{tab:all_detections_B3}-\ref{tab:all_detections_B7}.
The fits are reported in Tables
\ref{tab:NaCl_salt_lines}-\ref{tab:41K37Cl_salt_lines}.  Some of these
measurements (e.g., of the KCl v=5 45-44 line) are discrepantly
high, suggesting that they are blends with other species as denoted in 
Tables \ref{tab:all_detections_B3}-\ref{tab:all_detections_B7}.



\section{Discussion}
NaCl and KCl have only been seen in a small handful of other sources, all of
which were moderately high-mass evolved stars blowing off their envelopes.  The
known detections include the carbon star envelope IRC +10216
\citep{Cernicharo1987a,Agundez2012a}, the post-AGB star envelope CRL 2688
\citep{Highberger2003a}, and the oxygen-rich evolved star envelopes of VY Canis
Majoris and IK Tauri \citep{Milam2007a}. To our knowledge, \sourcei is only the
fifth astronomical source in which KCl and NaCl have been detected.
The only previous detection of vibrationally excited NaCl was of the v=1
line in IRC+10216 \citep{Quintana-Lacaci2016a}, 
% (\citealt{Agundez2012a} presented the most systematic study of salt lines, 
%which only included detections of v=0 rotational transitions),
while \sourcei exhibits clear emission up to v=6 in both NaCl and KCl.

In these evolved stars, v=0 rotational transitions of NaCl were observed
with rotational temperatures ${T_{rot}\sim70-100}$ K (IK Tau, VY CMa), inferred
local temperatures of $\sim200$ K in shocks (CRL 2688), and $T\approx700$ K  in
an expanding shell (IRC+10216).  Below, we attempt to determine the physical
conditions in the \sourcei salt emission regions.
% Carbon star envelope IRC +10216, d ~ 170 pc
% CRL2688 post-AGB * \citep{Highberger2003a} ~ 1 kpc
% IK Tauri, VY Canis Majoris \citep{Milam2007a}

\subsection{Where is the emission coming from?}
In \sourcei, the salt lines originate from the surface layers of the disk (the Band 7
lines exhibit NaCl emission peaks at $\pm0.032\arcsec=13$ AU above the disk),
% \citet{Ginsburg2018b} showed that these species come only from the disk's
% immediate surroundings, 
unlike SiO and \water that both exhibit high vertical extents consistent with
outflow \citep{Ginsburg2018b}.  No emission in the salt lines is observed
toward the optically thick continuum in the disk midplane (Figure
\ref{fig:spatial}).  Although the carriers of the lines were unassigned at the time, \citet{Ginsburg2018b} modeled the line emission now assigned to these salts as a
truncated Keplerian disk, finding that the radial distribution of the emission
has an inner cutoff $\approx35-40$ AU and an outer cutoff $\approx55-60$~AU.

% ((constants.sigma_sb * 4 * np.pi * (35*u.au)**2 / (1e4*u.L_sun))**(-1/4)).decompose()
% <Quantity 665.33595877 K>
% ((constants.sigma_sb * 4 * np.pi * (60*u.au)**2 / (1e4*u.L_sun))**(-1/4)).decompose()
% <Quantity 508.15873227 K>
The equilibrium temperature in this radius range can be computed assuming the central
source has a luminosity $L=10^4 \lsun = 4 \pi r^2 \sigma_{SB} T^4 $, where we
solve for temperature with $r$ at the inner and outer radius, giving $T_{eq} =
508-665$ K for $r=35-60$ AU, assuming the disk intercepts all of the starlight 
at this radius.  More realistically, providing the opposite limiting case,
$T_{eq}=120-185$ K for a flat disk \citep{Chiang1997a}.  These cooler temperatures
are consistent with the observed brightness temperatures in the outer
part of the continuum disk (in Figure \ref{fig:spatial}, the NaCl emission
begins just beyond the $T=300$ K contour), although the temperature in the inner
portion of the disk reaches $\gtrsim500$ K.

\begin{figure*}[!htp]
\includegraphics[scale=1,width=2.25in]{figures/OrionSourceI_NaClv=3_7-6_robust0.5.maskedclarkclean10000_medsub_K_peak_offset_contours.pdf}
\includegraphics[scale=1,width=2.25in]{figures/OrionSourceI_NaClv=1_18-17_robust0.5.maskedclarkclean10000_medsub_K_peak_offset_contours.pdf}
\includegraphics[scale=1,width=2.25in]{figures/OrionSourceI_NaClv=2_26-25_robust0.5.maskedclarkclean10000_medsub_K_peak_offset_contours.pdf}
\caption{Peak intensity images of NaCl lines in each band showing the spatial
distribution of the emission.  From left to right, the lines are: 87.26 GHz
NaCl v=3 J=7-6, 232.51 GHz NaCl v=1 J=18-17, and 333.01 GHz NaCl v=2 J=26-25.
These are the brightest uncontaminated lines in each of their observing bands.
The red contours show the Band 6 226 GHz continuum at levels of 50, 300, and 500 K.
White contours are shown at 50, 100, and 150 K.
The red and blue ellipses show the beams for the continuum and the line emission,
respectively.  The beam sizes are
$0.10 \arcsec \times 0.08 \arcsec$, PA $40.6 \degrees$,
$0.043 \arcsec \times 0.034 \arcsec$, PA $-87.4 \degrees$, and
$0.029 \arcsec \times 0.022 \arcsec$, PA $-54.6 \degrees$, respectively.
}
\label{fig:spatial}
\end{figure*}

We also compare the salt emission locations to the SiO emission regions.
Figure \ref{fig:sioonnacl} shows the \mbox{SiO v=0 J=8-7} line, which is
thermally excited (it shows no sign of masing at any velocity) and traces the
outflow.  The \mbox{SiO v=5 J=8-7} line, with an upper state energy level of
8747 K, is also shown.  It clearly traces a smaller radius than the NaCl line,
but a similar height in the disk.  By contrast, the \mbox{SiO v=0 J=8-7}
emission, with $E_U=75$ K, starts above the vertical centroid of the NaCl line.
These spatial anticorrelations suggest that the difference between the
molecules' emission patterns is driven partly by excitation, since the salt
lines have upper state energy levels intermediate between the SiO v=0 and v=5
states and trace an intermediate region.

\bam{This reads
as a result, not as discussion, as there's no conclusion drawn here.  What does
this result tell us?  Seems like we're saying that the production method for
SiO and NaCl/KCl probably aren't the same - they aren't co-located, so it's
probably not shock/outflow-induced for NaCl/KCl?  This runs contrary to what is
said later on in the discussion, however.}
\ag{Added a conclusion.}


\begin{figure*}[!htp]
\includegraphics[scale=1,width=4in]{figures/SiO_8-7_on_NaClv=2_26-25.pdf}
\caption{Peak intensity map of SiO v=0 J=8-7 (gray scale), NaCl v=2 J=26-25
(red contour; 200 K) and {SiO v=5 J=8-7} (blue contours; 300, 400, 500, and 600
K).  The blue contours in the upper-left come from a blend with a different
line.
}
\label{fig:sioonnacl}
\end{figure*}


% This estimate provides the expected temperature of
% gas at the surface of a passive disk in this radius range. {\color{red}We could
% obtain more realistic temperature estimates using the \citet{Chiang1997a}
% approximations for a flat or flared disk.}

\subsection{Excitation}
The detection of vibrationally excited transitions from v=0 to v=6 at similar
brightness suggests that the vibrational excitation temperature $T_{ex,vib}$ of
the molecules is high.  However, we see sharply decreasing populations in
the higher rotational levels, indicating a much lower $T_{ex,rot}$.
% Since the
% molecules are not characterized by a single $T_{ex}$, they may be excited by
% some non-thermal mechanism.

% We detect highly excited transitions of NaCl and KCl, many with Einstein A
% coefficients in the range $\sim0.01-0.1$ s$^{-1}$, suggesting they are
% radiatively rather than collisionally excited.

% These high excitation lines
% are unlikely to be excited by collisions, since the gas would have to be hot
% enough to excite molecules into states with $E_U > 2000$ K and would therefore
% likely be hot enough to dissociate the molecules {\color{red} It would be nice
% to back this up with numbers}.  Alternately, at high densities and more
% moderate temperatures, the salts would likely be absorbed into dust grains.

% We calculate the level population as
% \begin{equation}
%     N_u = \frac{8 \pi \nu k_B}{g_u c^2 h A_{ul}} \int T_A d\nu
% %    nline = 8 * np.pi * freq * constants.k_B / constants.h / Aul / constants.c**2
% %    # term2 = np.exp(-constants.h*freq/(constants.k_B*Tex)) -1
% %    # term2 -> kt / hnu
% %    # kelvin-hertz
% %    Khz = (kkms * (freq/constants.c)).to(u.K * u.MHz)
% %    return (nline * Khz / degeneracies).to(u.cm**-2)
% \end{equation}
% {\color{red} (this is mostly a sanity check)}

Within each vibrational state for which we observed several transitions, we attempt
to measure a rotational temperature.  For the KCl v=0 state, in which we have
detected 4 rotational transitions from $30 < E_U < 400$ K, the best fit
is $T_{rot}\sim105$ K (Figure
\ref{fig:rotationdiagrams}).  At such a low temperature, the populations of the
vibrationally excited states should be effectively zero.  

Similarly, for each rotational transition observed from multiple vibrational
levels, we measure a vibrational temperature.  We find values that range from
1000-5000 K, although these fits are subject to very large
uncertainties.
% the fits are consistent with any temperature within this range.
% (and, statistically but not physically, they are consistent with arbitrarily
% large excitation temperatures).  
There are also statistically significant
outliers from the single-temperature model, suggesting that the level
populations may not be well-characterized by a single temperature.

% The rotational temperatures are reasonably consistent with the lower bound
% disk temperatures at this radius, hinting that the salt emission lines are
% coming from somewhere below the surface of the disk.  If this is the case,
% there are likely to be few optical photons reaching the salt molecules,
% since those are readily absorbed by the small grains in the upper layers.


% Since the highest clearly detected energy state detected has $E_U\sim3000$ K
% (see Table \ref{tab:all_detections_B3}; the v=6 lines of NaCl have three
% J-transitions detected), the radiation field likely turns over at around this
% temperature, implying the presence of a slightly cooler radiation field than
% implied in \citet{Testi2010a}, but still much warmer than the equilibrium temperature
% of the disk at this radius.  
% However, it is also possible that the higher-excitation lines are simply too faint
% to be detected at our $\sim0.4$ mJy/\kms RMS limit.

% \subsubsection{Line Excitation Models}
Below, we explore several mechanisms for the observed line excitation:

\par{\textbf{(1) Collisional excitation by \hh.}} 
%
% \par{\textbf{(1) Collisional excitation sets the rotational temperatures, radiation
% sets the vibration:}}
% In Figure \ref{fig:rotationdiagrams}, we show that it is possible to acquire
% reasonable fits to the rotational transitions within a single vibrational state,
% obtaining $T_{ex,rot}\approx100-150$ K.  If we assume that the apparent consistency
% with a rotational temperature similar to the dust continuum temperature implies
% the rotational transitions are in local thermodynamic equilibrium with gas at 
% that temperature, we can infer a lower limit on the local density.
% However, we note that there may be unresolved excitation structure within our
% observations such that a single-zone, single temjperature model does not apply.
%
\citet{Quintana-Lacaci2016a} calculated collision rates for NaCl with He and
provided a Fortran code to compute those rates.  Typical collision rate coefficients for
\hbox{$\Delta$J=1} or \hbox{$\Delta$J=2} transitions are $10^{-10}$~cm$^3$ \pers.  Making the usual
assumption that the collision rates with \hh are similar to those for He, we used
RADEX \citep{van-der-Tak2007a} to examine the excitation \footnote{The
RADEX-compatible molecular data
file incorporating these coefficients is available from
\url{https://github.com/keflavich/SaltyDisk/blob/\githash/nacl.dat}}.
For T=100 K, $\delta v=1$ \kms \perpc, $N_{NaCl}=10^{12}$ \persc,
\footnote{\rlp{presumably,
the fractional abundance doesn't matter when one is computing critical density;
giving this value will create confusion; the linewidth also is questionable}
\ag{The fractional abundance relative to the collision partner does matter,
and the velocity gradient affects the optical depth.  These are fiducial,
poorly-justified parameters, but I think they're a good first guess.}
}
the critical density is $n_{cr}\sim10^8$ \percc for the $v=0$
states and $n_{cr}\sim10^{12}$ for the $v>=1$ states, where $v$ is the
vibrational state quantum number.  These high critical densities effectively
rule out collisional excitation producing vibrationally excited states,
since at such high densities, the rotational $v=0$ states would necessarily
be thermalized to the same temperature as the vibrational states, contrary
to what we observe.





\par{\textbf{(2) Collisional excitation by electrons.}} 
Because NaCl and KCl are extremely polar molecules, with dipole moments of 9.1
and 10.3 Debye, respectively \citep{Barton2014a}, collisions with electrons
could also be important.  Using the approximations from \citet{Dickinson1975a},
we find that collision rates between both NaCl and KCl and electrons in the
J=5-50 v=0 states are $C\approx10^{-5}$ cm$^{3}$ \pers.  Thus, electrons are
$10^5$ times as efficient at exciting these molecules than \hh. In typical
low-mass disks, ionization fractions $X_e \geq 10^{-4}$ occur in the upper
layers of the disk, where some UV radiation is able to ionize H and C
\citep{Bergin2007a}.  Collisions with electrons would dominate the excitation
of the salt lines in such regions.  However, deeper into the disks only X-rays
and cosmic rays can penetrate, resulting in fractional ionizations $X_e <
10^{-6}$, too small for electron collisions to be relevant. 

The low rotational temperatures that we infer seem to rule out both \hh and
electron collisional excitation of excited vibrational states, since a
collision rate high enough to excite these vibrational levels would quickly
thermalize their rotational ladders at high temperatures.


\par{\textbf{(3) Infrared excitation.}} 
The observed emission from vibrationally excited states suggests the
presence of a significant radiation field at 25-45 \um, which covers the range
of $\Delta v=1$ rovibrational transitions with $v\leq6$ (for NaCl, the range is
25-35 \um, for KCl, it is 35-45 \um).  Since the selection rules for these
transitions require that $\Delta v=1$ \footnote{Einstein A-values for the 
$\Delta v=2$ transitions are $\sim100$ times smaller than for the $\Delta v=1$
transitions, for which $A\sim1$ \pers.}, the radiation density must be high
enough to maintain a large population in the $v=1,
2, 3, 4, 5$ states such that some fraction of photons are able to excite
molecules to still higher states.  Such a strong radiation field would
be expected to thermalize the rotational ladders within each vibrational
state to the same high temperature, which is not observed.


\par{\textbf{(4) Ultraviolet excitation.}}
% In this model, collisions and infrared radiation can be ignored.  Instead,
% higher-energy photons are responsible for the excitation.  In order to directly
% excite the higher observed vibrationally excited states ($v\geq6$), 5 micron or
% shorter-wavelength photons are required.
%
% A model in which the excitation is driven by the radiation field at 4-10 microns,
% i.e., directly exciting the v=0 to the v=3-6 excitation states for NaCl,
% can be ruled out on the basis of the low interaction cross-sections of these
% transitions: they are highly forbidden, with transition probabilities $\sim10$
% orders of magnitude lower than the $\Delta v=1$ transitions.
%
The high vibrational and low rotational temperatures of the
salt lines resembles the `sawtooth' pattern observed for the rovibrational
populations of \hh toward the Orion bar photodissociation region.  This
pattern is explained by ultraviolet excitation into higher electronic states,
followed by decay back to excited vibrational levels in the ground electronic
state \citep{Kaplan2017a}.  \hh rotational populations in the ground vibrational
state are thermalized by collisions; UV excitation transposes this pattern into
higher vibrational states.
% without the $\Delta v=1$ restriction that applies to infrared excitation.

% Ultraviolet excitation into higher electronic states could explain the observed
% vibrationally excited emission.  Assuming that electronic states deexcite into
% random vibrational states, then trigger a subsequent cascade into the ground
% rotational and vibrational states, each UV photon would be responsible for emission
% spanning the full energy range.

Unfortunately, there are several problems applying this model to the salt lines.
First, we find an abrupt decline in the populations of $v > 6$ vibrational states,
with weak or ambiguous detections of NaCl in the v=7 and v=8 levels, and
nondetections for $v > 9$.  Unless there is
some selection effect in the electronic de-excitation disfavoring higher vibrational
states, it is difficult to explain this cutoff.  
Second, the binding energies of NaCl and KCl are only a few eV  \rlp{exact values here?}. 
they may therefore be dissociated rather than excited by UV photons.
Third, there may not be any UV radiation in the outer disk where these lines
are detected.  \sourcei has no electron-proton free-free emission
\citep{Plambeck2013a}, so a strong $\lambda<912$ \AA\ UV field is ruled out.  While UV
emission may be produced in shocks in the outflow, it is unclear whether
it could penetrate into the disk where
the salt emission is observed.  Even a light haze of dust in the upper layers of
the disk would likely shield the molecules from such illumination.

We conclude that there is no simple model that explains the observed pattern of
low rotational and high vibrational temperature.

% \textbf{(4) The rovibrational transitions are optically thick up to
% v$\sim$6, while the rotational transitions remain optically thin.}

% In this scenario, we assume that the salt molecules are all in an `atmosphere'
% above the disk.  While the disk is optically thick in the 25-45 \um continuum,
% this `atmosphere' is optically thin, such that the lines are not strictly in
% equilibrium with the radiation field.  However, the lowest rovibrational
% transitions are optically thick and therefore do reach equilibrium with the
% local radiation field set primarily by the disk.  The millimeter lines remain
% optically thin \rlp{I don't understand how this is possible; wouldn't the
% pure rotational levels within the lower vibrational states also be thermalized
% by the IR photons, hence end up at much higher temperature?}, so they remain collisionally rather than 
% radiatively excited.
% The higher rovibrational transitions are also optically thin, so the $v>6$
% states remain thin.  

% This scenario could imply that any other molecules in the disk atmosphere,
% such as CS or CO if they are present, would have large populations in their
% vibrationally excited states.

% It is not clear whether this model can explain the observed lines.  Modeling
% of the radiation field with realistic geometry for the disk (which provides
% an undiluted 100-500 K background) and the central source (which supplies a highly
% diluted $\sim4000$ K illumination) is needed.  Without including a background
% radiation field, this scenario does not appear to work, as the rotational
% transitions tend to be optically thicker than the vibrational if only collisional
% processes are active.

% {\color{red} todo: compute the column density needed to achieve $\tau=1$ in the
% v=1-0 J=10-9 transition, or similar}
%This scenario sounds enticing, but I'm not sure it will pass numerical checks.
%It may be that the columns required to get the rovibrational states to be
%optically thick would also drive the rotational states to be, which would be
%inconsistent with the observations.

% \textbf{(4) Collisions with electrons}
% Collisions with electrons are much more efficient than collisions with \hh.
% Using the approximations from \citet{Dickinson1975a}, we find that collision
% rates between both NaCl and KCl and electrons in the J=5-50 v=0 states are
% $C\approx10^{-5}$ cm$^{3}$ \pers.  These rates imply a critical electron
% abundance $X_e = C(\hh)/C(e^-) \approx 10^{-5}$, where we  assume
% $C(\hh)\approx10^{-10}$ cm$^3$ \pers, close to the typical value for
% the low-lying rotational transitions.  In other words, electrons are
% $10^5\times$ more efficient at exciting these molecules than \hh.

% The electron abundance may be higher than the critical value.  In typical
% low-mass disks, electron abundances $X_e \geq 10^4$ occur in the upper layers
% of the disk, where some UV radiation is able to ionize H and C to H+ and C+
% \citep{Bergin2007a}.  However, deeper into these disks, only X-rays and cosmic
% rays can penetrate to ionize neutrals, resulting in abundances too small to be
% relevant ($X_e < 10^{-6}$).  The importance of this scenario is critically
% sensitive to the ionization rate in the salt-bearing portion of the disk, which
% is uncertain.

% \ag{Two questions:
% (1) Is the electron temperature going to be different from the
% gas temperature?  I assume so, probably much higher!
% (2) What do electrons do in terms of vibrational excitation?
% The \citet{Dickinson1975a} work is focused entirely on rotational
% excitation.
% }
% 
% \bam{At least in laboratory plasmas, the vibrational temperature of many
% molecules is directly coupled to the electron temperature.  Different density
% environments, though...}


%\textbf{Text from earlier}
%
%%{\color{red} This is a note to be incorporated somewhere:}
%The $\Delta v=1$ rovibrational transitions of KCl are all around 36.1-38 \um,
%while the NaCl transitions are in the range 27.6-30 \um.  If there is a strong
%radiation field at these wavelengths, it may pump the lines into progressively
%higher vibrational states.

% More interesting for constraining the exciting source, though, are the v=0 to
% v=1 through v=6 lines.
% For example, the KCl v=6-0 J=1-0 line is at 6.13 \um and the NaCl v=7-0 J=1-0
% line is at 4.07 \um.  \bam{These would be wickedly forbidden transitions.
% Allowed vibrational transitions have $\Delta$V = $\pm$1.  Are there better
% candidates that would follow that?  Say a v$_6$ - v$_5$ or something?}
% \ag{Right, these are too forbidden to be plausible: this paragraph will be
% removed; we need to find another way to excite the molecules up to these states.}
% Assuming
% the population of these lines are created
% from excitation from the ground state, they require a radiation field with
% significant emission in the 4-6 \um range.
% A blackbody with f $T\approx500-750$ K
% % ($T\approx b / \lambda$, where $b\approx2900$ \um K is Wien's displacement constant)
% peaks in this range.
% [MCHW : consistent with Tb observed in the high resolution images]

% A plausible scenario is that the hot radiation field from the central star,
% with $T_*\sim4000$ K \citep{Testi2010a}, excites molecules to high vibrational
% excitation states, from which they cascade through rovibrational transitions
% down to the v=0 state, providing a relatively uniform level population
% across the excited vibrational states instead of a thermal level population.
% However, a 4000 K blackbody should produce most of its flux around 1 \um,
% which should result in much higher excitation than we observe, suggesting
% that the starlight is extincted before reaching the salts.
% 
% The densities required to produce equilibrium populations within the individual
% vibrationally excited states are very high.  The Einstein A coefficients of the
% $\Delta v=1$ transitions are of order $\sim1$ \pers.  If we assume the
% collision rates for the salts are similar to SiS, the most similar-mass
% two-atom molecule available from
% \url{http://home.strw.leidenuniv.nl/~moldata/datafiles}, with values
% $\sim10^{-12}-10^{-10}$ cm$^{3}$ \pers, densities exceeding $n>10^{10}$ \percc
% (possibly $n>10^{12}$ \percc, depending on which collisional coefficients are
% used) are required to have collisions dominate over radiation. 
% 
% \ag{However, I think there's a problem with the above logic: if the densities
% really are that high, i.e., high enough that there are typically several collisions
% per spontaneous emission, then we should not expect to see any spontaneous emission.
% So instead, we're left with an upper limit density of $n<10^{10}-10^{12}$ \percc,
% but no clear explanation for the apparent consistent rotational temperatures.
% }


% Either
% the higher-excitation lines are too faint {\color{red} do we expect this?}
% or the salt emission is coming from a region of the disk that is shielded
% behind an extinction layer.   Since the salts would likely be incorporated into
% grains {\color{red}can we back this?} if a large population of grains
% were present, it is not clear that such a layer can exist.



% We plot and fit rotational diagrams to measure the abundance and temperature of
% these molecules.  However, the results are ambiguous: for each vibrational
% state with more than one detected transition, we are able to fit an excitation
% temperature, but the abundances of the vibrationally excited states are much
% higher than the ground state.  This distribution implies that the lines are
% radiatively excited.
%
% The excitation temperatures derived from the rotational diagram fits are in the
% range 50-100 K.  The
% radiation temperature within 50 AU of a $10^4$ \lsun star should be $T_R>500$ K,
% so how do we get lower temperatures?
% {\color{red} This is a place for some discussion: is it possible that the gas
% is cooler and the transitions we're seeing are affected by collisions?
% This is the process invoked in \citet{Kaplan2017a}, but it doesn't make sense here:
% it requires the gas to be below the radiation temperature.  Also, the critical density
% of these transitions is likely to be absurdly high, given their A-values in the 0.1s
% range.}



\subsection{Why are these salts present?}

Sodium and potassium are rarely observed in the dense molecular interstellar
medium.  They are usually assumed to be rapidly incorporated into dust grains
after being ejected from dying stars \citep[e.g.,][]{Milam2007a}.  Since we
observe NaCl and KCl in the atmosphere of the disk, it is clear that
there is a zone where either dust has not (yet?) formed or
where dust is destroyed and returned to the gas phase.  We argue that the dust
must be destroyed almost immediately as it is launched into the disk-driven
outflow \citep{Hirota2017a}.
% As above, we evaluate several possibilities to explain the presence of
% these molecules.

% I really hope someone takes this too seriously...
%Most likely, the lack of salt in the outflow is because it's dissolved into
%the \water.  The outflow is a beach.


% The HCl+ ion has several transitions in our band 3, but none are detected.
% This could be an excitation issue, as these lines have upper state energies
% In the range 1200-1400 K, but since similar excitation lines of NaCl are
% Seen, it is likely that HCl+ has a lower abundance...
% Is this in any way useful or constraining?
% The real question is, why don't we detect any hydrogen-containing molecules
% In the disk?

% If we assume the column density for the v=0 state represents the total
% column of KCl - which is a fairly bad assumption - we can infer the relative abundance.
% We assume the disk has a column density

% {\color{red} What can
% we say about the dust destruction/sublimation zone?  I need to do some disk
% research; there must be a dust sublimation zone at the surfaces of disks in
% addition to the radial one.}


In this scenario,  NaCl and KCl molecules are sputtered from the grains.  The
most plausible means to achieve the energies required to sputter grains is with
strong shocks \citep{Schilke1997a}.  In
\sourcei, it seems unlikely that there would be very strong shocks in the disk,
but there are high-velocity shocks throughout the outflow. Strong SiO emission
is observed in the inner part of the outflow (i.e., within $<100$ AU of
\sourcei), suggesting that grains are efficiently destroyed there.
Without knowing the grain structure, however, it is unclear whether high-energy
particles capable of destroying the grain cores are present, or if instead
the salts are being released from evaporation of the grain mantles.

Since the salts are detected close to but above the disk midplane, the
grains must be sputtered very rapidly after being launched from the disk surface.
The lack of salt and SiO emission in the disk midplane may be because these
molecules are not in the gas phase at all in the midplane, although radiative
transfer can also easily explain their non-detection (i.e., the continuum
background is the same intensity as the emission lines).


In this model, the different morphology of the SiO and NaCl lines in the disk
and outflow is driven mostly by excitation, since both SiO and the salts are
in the gas phase.  However, excitation alone cannot explain this difference,
since there are low-excitation lines of KCl and NaCl observed that do not
show any vertical extent.  Most likely, the salts are depleting back onto
grains or reacting and forming other molecules more rapidly than SiO.
% \bam{I can't buy this, actually.  That's a *huge*
% difference to be due just to excitation.  Perhaps the salts are just depleting
% out back onto the grains (or reacting away) faster than SiO $\rightarrow$
% SiO$_2$?  This is perhaps more probable to me if we work with the
% oxygen-depleted route, since there's less oxygen to make the SiO$_2$?  This is
% all just spit-balling, though.}
% This hypothesis is plausible, since the
% SiO v=0 J=8-7 transition shown in Figure \ref{fig:sioonnacl} has $A_{8,7} =
% 2.2030\ee{-3}$ \pers, while the NaCl v=2 J=26-25 transition has $A_{26,25} =
% 1.8242\ee{-2}$ \pers, and their collision rates are nearly the same (assuming SiS and NaCl have
% identical collision rates, as above, and that vibrational excitation does not
% change the rates), implying that the critical density of the
% NaCl transition is $\sim10\times$ higher.

While NaCl and KCl are clearly detected here, there are also several transitions of
AlCl and AlF in our observational bands that were not detected.
\citet{Cernicharo1987a} detected these transitions at a comparable brightness to NaCl
and KCl in IRC+10216.  The lack of AlCl may be because \sourcei's disk is
oxygen-rich, and aluminum is locked into AlOH, as suggested in the
\citet{Cherchneff2012a} model.  Indeed, the tentative detection of AlO
in our spectra supports this conclusion.


We consider an alternative hypothesis: the salts are observed because they are
presently forming directly in the gas phase.  Potassium and sodium have similar
first ionization potentials at 4.34 and 5.14 eV, respectively.  These atoms
would be nearly fully ionized between 1500-2000 K at a density of $10^6$
\percc (Aluminum has a higher first ionization potential of 5.99 eV, and so is ionized
at a few hundred K higher temperatures than sodium).  These ions react with HCl
to form NaCl and KCl.  However, at present, only the inner $<20$ AU region,
which is currently unresolved, clearly reaches
such high temperatures.  Either material in that inner region is being
transported outward, which seems unlikely since there is a bulk outflow lifting
material off of the disk, or the disk was previously substantially warmer.
Under the dynamical formation scenario of the disk, in which it is the remnant
of a previously larger disk from the \sourcei-BN interaction
\citep{Bally2017a,Luhman2017a}, a great deal of kinetic energy was released as
the disk re-settled into its current apparently smooth state.  This scenario
implies that the observable state of NaCl and KCl is very short lived, as these
molecules are in the process of depleting onto grains.
While plausible, there are many processes that need to coincide perfectly
in this scenario (disk temperature history, gas density, gas-phase formation
rates), so we favor the dust destruction hypothesis.


%(1) the dust is totally vaporized, so the salts are in the gas phase.  This is
%pretty implausible, since it presumably requires temperatures >1000 K,
%and the highest excitation line can be excited by 4 \um photons.
%Also, if we trust the rotational temperatures, there has to be dust
%shielding to keep the salt-emitting layer cool.  If we trust those temperatures,
%it is hard to understand why the molecules haven't condensed into grains
%already, unless the dust chemical equilibrium timescale is long.  But, because
%RCB stars exist, I think the dust formation timescale is very tiny, more like
%months or years?

%\subsubsection{Alternative 1: The salts are just now beginning to fr
%(2) the salts are just forming now as they’re released from the star.  This
%doesn’t work at all if the star is accreting, BUT if the collision hypothesis
%is correct, we might see tons of stellar interior products mixed into the
%disk’s atmosphere.  This scenario requires the adsorption timescale for the
%salts to be $>500$ yr, which again seems to contradict the existence of RCB
%stars that rapidly form dust at probably comparable or lower densities.
%Also, I believe Na, K, and Cl are supernova products, not stellar nucleosynthesis
%products, but I am not sure of that!
%
%\bam{(3) Stupid idea: the salts are ejected from the star, do condense out, and
%then are being ejected again in the outer edges of the disk as there is some
%sort of turbulent mixing there grinding the dust back up?  If that's the case
%though, wouldn't we expect to see SiO there as well? hm...}
%\ag{Since these metals are SN products, both in this case and in case (2),
%we're just talking about putting gas-phase materials out that already
%had the same metallicity as the star.  So I think (3) is best simplified
%as just: ``the dust is naturally `ground up' in the disk''.}

\subsection{Isotopologue abundances}
With several transitions of each of the observed isotopologues, we can in
principle measure the relative abundance of the various isotopes.  However,
because of the uncertainties in the excitation above, these measurements should
be taken with a grain of salt.

The $^{37}$Cl abundance can be measured by comparing the intensity of the
NaCl and Na$^{37}$Cl v=3 J=7-6, v=4 J=8-7 and J=7-6, and v=5 J=8-7 lines.
The ratio $^{35}$/$^{37}$Cl is $r=1.4\pm0.4$.  The KCl v=0 J=45-44 line
gives a $^{35}/^{37}$Cl ratio $r=0.8$.  \citet{Agundez2012a} measured
this ratio to be $r=2.9\pm0.3$ in IRC+10216.  Our measurement is somewhat lower,
but should be regarded as consistent at least until we can obtain more
accurate column density measurements.

We only have one commonly observed $^{41}$KCl / $^{39}$KCl line, v=0 J=13-12,
which has a ratio $^{39}/^{41}$ $r=2.4$.  \ag{Have there been any measurements
of this ratio?}


\subsection{Future prospects}
The detection of these refractory species in the gas phase in the atmosphere
of \sourcei's disk hints that other refractory molecules may be present, 
which could enable their first detection in the ISM (e.g., such rare species as
FeO and FeS), and may enable direct measurments of the metallicity in
star-forming gas.
It remains unclear whether \sourcei is a unique source or is representative
of the class of high-mass protostars with disks; if the latter, these lines
and potentially many others can be used to probe the radiation environment of 
extremely embedded high-mass protostars.  

The detection of these species with substantial population in vibrationally
excited states highlights the need for laboratory work to measure such
transitions in known astronomical molecular species. Indeed, prior work on
vibrational excitation in modeling and assigned vibrationally excited ethyl
cyanide in ALMA observations of Orion-KL revealed that even when catalogs for
these states are available, they are often incomplete \citep{Fortman2012a}.

\begin{figure*}[!htp]
\includegraphics[scale=1,width=3.5in]{figures/KCl_rotational_diagrams.pdf}
\includegraphics[scale=1,width=3.5in]{figures/NaCl_rotational_diagrams.pdf}
\caption{Rotational energy diagrams for the KCl and NaCl lines.  While each
vibrational state can internally be explained reasonably well by a single
consistent rotation temperature in the range $T_{rot}\sim50-150$ K, the population
distribution between vibrational excitation states cannot.}
\label{fig:rotationdiagrams}
\end{figure*}

\begin{table*}[htp]
\centering
\caption{NaCl Lines}
\begin{tabular}{ccccccc}
\label{tab:NaCl_salt_lines}
 J$_u$ & J$_l$ & Frequency & Velocity & Width & Amplitude & E$_U$ \\
  &  & $\mathrm{GHz}$ & $\mathrm{km\,s^{-1}}$ & $\mathrm{km\,s^{-1}}$ & $\mathrm{K}$ & $\mathrm{K}$ \\
\hline
&\vspace{-0.75em}\\
\multicolumn{7}{c}{$v = 0$} \\
\vspace{-0.75em}\\
0 & 45 & 44 & 335.05221 & 6.2 (0.4) & 6.4 (0.4) & 12.1 (0.7) & 370.4 \\
 18 & 17 & 229.24720 & 3.3 (0.1) & 4.7 (0.1) & 21.4 (0.4) & 104.6 \\
 7 & 6 & 89.22066 & 2.5 (0.4) & 5.0 (0.4) & 18.2 (1.2) & 17.1 \\
&\vspace{-0.75em}\\
\multicolumn{7}{c}{$v = 2$} \\
\vspace{-0.75em}\\
 31 & 30 & 229.68076 & 4.2 (0.6) & 13.1 (1.0) & 9.6 (0.3) & 962.5 \\
 13 & 12 & 98.70507 & 4.3 (0.9) & 7.2 (1.0) & 13.6 (1.4) & 828.3 \\
 18 & 17 & 230.77811 & 6.9 (0.1) & 6.0 (0.1) & 19.5 (0.4) & 1138.6 \\
&\vspace{-0.75em}\\
\multicolumn{7}{c}{$v = 5$} \\
\vspace{-0.75em}\\
 45 & 44 & 334.29723 & 2.7 (0.3) & 16.6 (0.6) & 42.3 (0.5) & 2332.2 \\
 31 & 30 & 230.72302 & 6.5 (0.4) & 4.3 (0.4) & 5.2 (0.5) & 2139.8 \\
&\vspace{-0.75em}\\
\multicolumn{7}{c}{$v = 6$} \\
\vspace{-0.75em}\\
 30 & 29 & 215.73578 & 7.3 (2.8) & 3.0 (2.8) & 1.9 (1.5) & 2473.0 \\
\hline
\end{tabular}

\par 
\end{table*}

\begin{table*}[htp]
\centering
\caption{Na$^{37}$Cl Lines}
\begin{tabular}{ccccccc}
\label{tab:Na37Cl_salt_lines}
 J$_u$ & J$_l$ & Frequency & Velocity & Width & Amplitude & E$_U$ \\
  &  & $\mathrm{GHz}$ & $\mathrm{km\,s^{-1}}$ & $\mathrm{km\,s^{-1}}$ & $\mathrm{K}$ & $\mathrm{K}$ \\
\hline
&\vspace{-0.75em}\\
\multicolumn{7}{c}{$v = 0$} \\
\vspace{-0.75em}\\
 7 & 6 & 89.22011 & 4.3 (0.4) & 5.0 (0.4) & 18.2 (1.2) & 17.1 \\
 18 & 17 & 229.24605 & 4.8 (0.1) & 4.7 (0.1) & 21.4 (0.4) & 104.6 \\
 17 & 16 & 214.93871 & 4.8 (0.4) & 4.6 (0.4) & 18.7 (1.2) & 607.0 \\
&\vspace{-0.75em}\\
\multicolumn{7}{c}{$v = 1$} \\
\vspace{-0.75em}\\
 8 & 7 & 101.21188 & 4.6 (0.3) & 5.7 (0.3) & 19.7 (0.8) & 536.0 \\
 8 & 7 & 100.46695 & 5.3 (0.4) & 7.9 (0.5) & 15.6 (0.7) & 1045.0 \\
&\vspace{-0.75em}\\
\multicolumn{7}{c}{$v = 2$} \\
\vspace{-0.75em}\\
 7 & 6 & 87.91232 & 4.3 (0.6) & 6.3 (0.6) & 13.1 (1.1) & 1040.1 \\
 7 & 6 & 87.26464 & 2.9 (1.0) & 4.5 (1.0) & 9.0 (1.7) & 1544.3 \\
&\vspace{-0.75em}\\
\multicolumn{7}{c}{$v = 3$} \\
\vspace{-0.75em}\\
 8 & 7 & 99.72675 & 3.7 (0.4) & 5.8 (0.4) & 14.0 (0.8) & 1549.1 \\
 7 & 6 & 86.62109 & -2.1 (1.1) & 8.1 (1.2) & 13.0 (1.4) & 2043.6 \\
&\vspace{-0.75em}\\
\multicolumn{7}{c}{$v = 4$} \\
\vspace{-0.75em}\\
 8 & 7 & 98.99127 & 4.6 (1.1) & 4.2 (1.1) & 8.2 (1.9) & 2048.4 \\
 27 & 26 & 333.45906 & 5.5 (0.5) & 4.8 (0.5) & 10.9 (1.0) & 2251.3 \\
 19 & 18 & 233.16920 & 5.2 (0.5) & 3.9 (0.5) & 7.4 (0.8) & 2633.6 \\
\hline
&\vspace{-0.75em}\\
\multicolumn{7}{c}{$v = 5$} \\
\vspace{-0.75em}\\
 7 & 6 & 85.98167 & 5.2 (1.2) & 4.9 (1.2) & 8.1 (1.7) & 2538.2 \\
\end{tabular}

\par 
\end{table*}

\begin{table*}[htp]
\centering
\caption{KCl Lines}
\begin{tabular}{ccccccc}
\label{tab:KCl_salt_lines}
 J$_u$ & J$_l$ & Frequency & Velocity & Width & Amplitude & E$_U$ \\
  &  & $\mathrm{GHz}$ & $\mathrm{km\,s^{-1}}$ & $\mathrm{km\,s^{-1}}$ & $\mathrm{K}$ & $\mathrm{K}$ \\
\hline
&\vspace{-0.75em}\\
\multicolumn{7}{c}{$v = 0$} \\
\vspace{-0.75em}\\
 28 & 27 & 215.00828 & 5.5 (0.6) & 4.3 (0.6) & 10.4 (1.3) & 149.7 \\
 30 & 29 & 230.32064 & 4.6 (0.2) & 4.8 (0.2) & 14.4 (0.4) & 171.4 \\
 45 & 44 & 344.82061 & 2.5 (1.1) & 4.9 (1.1) & 9.7 (1.8) & 381.2 \\
 13 & 12 & 99.92952 & 4.0 (0.3) & 4.8 (0.3) & 15.3 (0.9) & 33.6 \\
&\vspace{-0.75em}\\
\multicolumn{7}{c}{$v = 1$} \\
\vspace{-0.75em}\\
 44 & 43 & 335.13396 & 3.3 (0.3) & 4.1 (0.3) & 14.1 (0.8) & 761.7 \\
&\vspace{-0.75em}\\
\multicolumn{7}{c}{$v = 2$} \\
\vspace{-0.75em}\\
 13 & 12 & 98.70595 & 2.0 (0.8) & 6.4 (0.8) & 14.2 (1.5) & 828.3 \\
&\vspace{-0.75em}\\
\multicolumn{7}{c}{$v = 3$} \\
\vspace{-0.75em}\\
 13 & 12 & 98.09753 & 6.1 (1.1) & 5.0 (1.1) & 9.2 (1.7) & 1220.6 \\
 29 & 28 & 218.57971 & 4.4 (0.4) & 3.3 (0.4) & 5.0 (0.6) & 1345.1 \\
&\vspace{-0.75em}\\
\multicolumn{7}{c}{$v = 4$} \\
\vspace{-0.75em}\\
 13 & 12 & 97.49133 & 8.0 (2.1) & 7.0 (2.4) & 6.1 (1.5) & 1609.5 \\
 29 & 28 & 217.22891 & -0.3 (0.6) & 3.2 (0.6) & 3.5 (0.6) & 1733.2 \\
&\vspace{-0.75em}\\
\multicolumn{7}{c}{$v = 5$} \\
\vspace{-0.75em}\\
 31 & 30 & 230.72399 & 5.3 (0.4) & 4.3 (0.4) & 5.2 (0.5) & 2139.8 \\
 29 & 28 & 215.88373 & 6.3 (2.1) & 4.3 (2.1) & 3.0 (1.3) & 2118.0 \\
&\vspace{-0.75em}\\
\multicolumn{7}{c}{$v = 6$} \\
\vspace{-0.75em}\\
 47 & 46 & 346.87489 & 0.7 (1.1) & 5.5 (1.2) & 4.4 (0.8) & 2745.3 \\
 29 & 28 & 214.54412 & 6.3 (1.9) & 4.4 (1.9) & 3.4 (1.3) & 2499.6 \\
\hline
\end{tabular}

\par 
\end{table*}

\begin{table*}[htp]
\centering
\caption{K$^{37}$Cl Lines}
\begin{tabular}{ccccccc}
\label{tab:K37Cl_salt_lines}
 J$_u$ & J$_l$ & Frequency & Velocity & Width & Amplitude & E$_U$ \\
  &  & $\mathrm{GHz}$ & $\mathrm{km\,s^{-1}}$ & $\mathrm{km\,s^{-1}}$ & $\mathrm{K}$ & $\mathrm{K}$ \\
\hline
&\vspace{-0.75em}\\
\multicolumn{7}{c}{$v = 0$} \\
\vspace{-0.75em}\\
 45 & 44 & 335.05221 & 6.2 (0.4) & 6.4 (0.4) & 12.1 (0.7) & 370.4 \\
&\vspace{-0.75em}\\
\multicolumn{7}{c}{$v = 1$} \\
\vspace{-0.75em}\\
 12 & 11 & 89.08229 & 5.2 (1.2) & 4.7 (1.2) & 5.5 (1.2) & 421.4 \\
 31 & 30 & 229.81801 & 6.5 (0.3) & 4.8 (0.3) & 7.2 (0.4) & 570.2 \\
\hline
&\vspace{-0.75em}\\
\multicolumn{7}{c}{$v = 6$} \\
\vspace{-0.75em}\\
 30 & 29 & 215.73578 & 7.3 (2.8) & 3.0 (2.8) & 1.9 (1.5) & 2473.0 \\
\end{tabular}

\par 
\end{table*}

\begin{table*}[htp]
\centering
\caption{$^{41}$KCl Lines}
\begin{tabular}{ccccccc}
\label{tab:41KCl_salt_lines}
 J$_u$ & J$_l$ & Frequency & Velocity & Width & Amplitude & E$_U$ \\
  &  & $\mathrm{GHz}$ & $\mathrm{km\,s^{-1}}$ & $\mathrm{km\,s^{-1}}$ & $\mathrm{K}$ & $\mathrm{K}$ \\
\hline
&\vspace{-0.75em}\\
\multicolumn{7}{c}{$v = 0$} \\
\vspace{-0.75em}\\
 29 & 28 & 217.54351 & 5.0 (0.7) & 6.5 (0.7) & 4.5 (0.4) & 156.7 \\
 13 & 12 & 97.62803 & 4.9 (2.5) & 10.0 (3.2) & 6.5 (1.3) & 32.8 \\
&\vspace{-0.75em}\\
\multicolumn{7}{c}{$v = 2$} \\
\vspace{-0.75em}\\
 12 & 11 & 89.03049 & 8.2 (0.0) & 0.2 (0.0) & 0.1 (1.6) & 813.8 \\
\hline
\end{tabular}

\par 
\end{table*}

\begin{table*}[htp]
\centering
\caption{$^{41}$K$^{37}$Cl Lines}
\begin{tabular}{ccccccc}
\label{tab:41K37Cl_salt_lines}
 J$_u$ & J$_l$ & Frequency & Velocity & Width & Amplitude & E$_U$ \\
  &  & $\mathrm{GHz}$ & $\mathrm{km\,s^{-1}}$ & $\mathrm{km\,s^{-1}}$ & $\mathrm{K}$ & $\mathrm{K}$ \\
\hline
\hline
&\vspace{-0.75em}\\
\multicolumn{7}{c}{$v = 7$} \\
\vspace{-0.75em}\\
7 & 48 & 47 & 334.32520 & -9.5 (0.0) & 7.1 (0.3) & 27.8 (0.8) & 3049.3 \\
\end{tabular}

\par 
\end{table*}



\section{Conclusions}
We have identified many transitions of NaCl, KCl, and their isotopologues in
the disk of \sourcei.  These lines trace material very near the surface of the
disk, providing a uniquely powerful probe of the disk kinematics and physical
conditions.

Despite the wide range of transitions observed, the excitation mechanism and
conditions remain uncertain.  Further observations of lower-energy vibrational
and rotational states of these molecules will help distinguish between
radiative and collisional excitation scenarios and will either provide
direct measurements of the density or the radiation field in the 30-60 AU
region around \sourcei.

The narrow vertical extent of the salt emission indicates that dust is
destroyed nearly immediately after being raised from the surface of the disk.
This hints that immediate dust destruction is an integral part of the outflow
driving process.

%These lines trace a rare physical phenomenon.  They are therefore an excellent tracer
%of this phenomenon, assuming they are detected


\acknowledgements
The National Radio Astronomy Observatory is a facility of the National Science
Foundation operated under cooperative agreement by Associated Universities,
Inc. The Green Bank Observatory is a facility of the National Science
Foundation operated under cooperative agreement by Associated Universities,
Inc. Support for B.A.M. was provided by NASA through Hubble Fellowship grant
\#HST-HF2-51396 awarded by the Space Telescope Science Institute, which is
operated by the Association of Universities for Research in Astronomy, Inc.,
for NASA, under contract NAS5-26555. 
This paper makes use of the following ALMA data: ADS/JAO.ALMA\#2016.1.00165.S
ALMA is a partnership of ESO (representing its member states), NSF (USA) and
NINS (Japan), together with NRC (Canada), MOST and ASIAA (Taiwan), and KASI
(Republic of Korea), in cooperation with the Republic of Chile. The Joint ALMA
Observatory is operated by ESO, AUI/NRAO and NAOJ.

\software{
The software used to make this version of the paper is available from github at
\url{https://github.com/keflavich/Orion_ALMA_2016.1.00165.S}
(\dataset[doi:10.5281/zenodo.1213350]{https://doi.org/10.5281/zenodo.1213350}) 
with hash \githash
(\gitdate).  The tools used include \texttt{spectral-cube}
\citep[][and \url{https://github.com/radio-astro-tools/spectral-cube}]{Ginsburg2018SpectralCube}
and
\texttt{radio-beam} 
\citep[][and \url{https://github.com/radio-astro-tools/radio-beam}]{Ginsburg2018RadioBeam}
from the
\texttt{radio-astro-tools} package
(\url{radio-astro-tools.github.io}), \texttt{astropy}
\citep{Astropy-Collaboration2013a}, \texttt{astroquery}
\citep[][and~\url{astroquery.readthedocs.io}]{Ginsburg2018Astroquery}
and \texttt{CASA} \citep{McMullin2007a}.
A script to produce the collision rate table file using the fortran code
provided by \citet{Quintana-Lacaci2016a} is available at
\url{https://github.com/keflavich/Orion_ALMA_2016.1.00165.S/blob/master/analysis/collision_rates_nacl.py}.
}


\input{solobib}

\end{document}
